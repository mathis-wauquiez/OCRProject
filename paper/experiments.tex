% ============================================================================
%  Experiments — Ablation Studies, Comparisons, and Evaluation Protocol
% ============================================================================
%
%  This file is designed to be \input{} into main.tex.
% ============================================================================

\section{Experiments}
\label{sec:experiments}

This section describes the evaluation protocol, ablation studies, comparative experiments, and the planned figures for the paper.


% ============================================================================
\subsection{Evaluation Protocol}
\label{sec:eval-protocol}

% ---------- Metrics ----------
\subsubsection{Metrics}
\label{sec:eval-metrics}

All metrics are computed against manually annotated reference labels (not OCR-derived labels).

\begin{table}[t]
    \centering
    \small
    \caption{Evaluation metrics and their interpretation.}
    \label{tab:metrics}
    \begin{tabular}{@{}lp{8.5cm}@{}}
        \toprule
        \textbf{Metric} & \textbf{Description} \\
        \midrule
        ARI~\cite{hubert1985comparing} & Chance-adjusted pairwise agreement; primary metric. Range $[-1, 1]$. \\
        NMI & Information-theoretic overlap; robust to different cluster counts. Range $[0, 1]$. \\
        V-measure~\cite{rosenberg2007v} & Harmonic mean of homogeneity (within-cluster purity) and completeness (all instances of a class in one cluster). Decomposes errors into splits and merges. \\
        Per-class F1 & Precision/recall per character class via Hungarian matching. Identifies problematic character pairs. \\
        Noise fraction & Fraction of characters classified as noise (HDBSCAN only). Should correlate with detection errors and damage. \\
        \bottomrule
    \end{tabular}
\end{table}

% ---------- Ground truth ----------
\subsubsection{Ground Truth Construction}
\label{sec:ground-truth}

At least 500 characters (covering all classes, including rare ones) from each corpus are manually annotated by two independent annotators.
Inter-annotator agreement is reported via Cohen's~$\kappa$.
Disagreements are resolved by consensus.
The annotated set is independent of the OCR system: annotators work from character crop images alone, without seeing OCR predictions.

% ---------- Statistical significance ----------
\subsubsection{Statistical Significance}
\label{sec:significance}

For all main comparisons (method A vs.\ method B):
\begin{itemize}[nosep]
    \item Mean and standard deviation over 5 bootstrap sub-samplings of the evaluation set.
    \item Paired Wilcoxon signed-rank test for per-character classification agreement.
    \item Cohen's $d$ effect size for ARI differences.
\end{itemize}


% ============================================================================
\subsection{Ablation Studies}
\label{sec:ablations}

Each ablation isolates one variable while holding all others at their default setting.
We report ARI, NMI, V-measure, and the number of clusters.

% ---------- Feature ablations ----------
\subsubsection{Feature Representation (A1--A4)}

\begin{table}[t]
    \centering
    \small
    \caption{Feature representation ablations.}
    \label{tab:ablation-features}
    \begin{tabular}{@{}llp{6.5cm}@{}}
        \toprule
        \textbf{ID} & \textbf{Variable} & \textbf{Variants} \\
        \midrule
        A1 & HOG cell size & $8 \times 8$, $12 \times 12$, $16 \times 16$, $24 \times 24$\,px \\
        A2 & Orientation bins & 4, 8, 12, 16, 24 unsigned bins \\
        A3 & Feature type & HOG (baseline), Shape Context, Learned (contrastive CNN), HOG + persistent homology pre-filter \\
        A4 & Dissimilarity & CEMD (baseline) vs.\ Sinkhorn OT on spatial-feature tensor \\
        \bottomrule
    \end{tabular}
\end{table}

% ---------- A contrario ablations ----------
\subsubsection{A Contrario Framework (A5--A7)}

\begin{table}[t]
    \centering
    \small
    \caption{A contrario framework ablations.}
    \label{tab:ablation-acontrario}
    \begin{tabular}{@{}llp{6.5cm}@{}}
        \toprule
        \textbf{ID} & \textbf{Variable} & \textbf{Variants} \\
        \midrule
        A5 & Background model & Na\"ive $\mathcal{H}_0$ (independent cells) vs.\ Empirical null (bootstrapped) vs.\ Gamma-fitted \\
        A6 & Gaussianity & Q-Q plots at $K = 4, 16, 36, 64$ cells; KS $p$-values \\
        A7 & NFA threshold $\varepsilon$ & $\{0.01, 0.1, 1, 10, 100\}$; plot ARI and $K$ vs.~$\varepsilon$ \\
        \bottomrule
    \end{tabular}
\end{table}

% ---------- Clustering ablations ----------
\subsubsection{Clustering Method (A8--A10)}

\begin{table}[t]
    \centering
    \small
    \caption{Clustering method ablations.}
    \label{tab:ablation-clustering}
    \begin{tabular}{@{}llp{6.5cm}@{}}
        \toprule
        \textbf{ID} & \textbf{Variable} & \textbf{Variants} \\
        \midrule
        A8 & Algorithm & Connected components, HDBSCAN, AP, Louvain, Leiden \\
        A9 & HDBSCAN \texttt{min\_cluster\_size} & $\{2, 3, 5, 10, 20\}$; report ARI, $K$, noise fraction \\
        A10 & HDBSCAN \texttt{min\_samples} & $\{1, 3, 5, 10, 15\}$ \\
        \bottomrule
    \end{tabular}
\end{table}

% ---------- Refinement ablations ----------
\subsubsection{Refinement (A11--A13)}

\begin{table}[t]
    \centering
    \small
    \caption{Refinement ablations.}
    \label{tab:ablation-refinement}
    \begin{tabular}{@{}llp{6.5cm}@{}}
        \toprule
        \textbf{ID} & \textbf{Variable} & \textbf{Variants} \\
        \midrule
        A11 & Refinement method & None, Current 3-stage, K-medoids split/merge, MRF (NFA+OCR), HDBSCAN (no refinement needed) \\
        A12 & MRF weight $\beta$ & $\{0.1, 0.5, 1.0, 2.0, 5.0\} \times (\text{median NLFA})^{-1}$ \\
        A13 & Stage ablation & Full pipeline, $-$Stage~1, $-$Stage~2, $-$Stage~3, no refinement \\
        \bottomrule
    \end{tabular}
\end{table}

% ---------- Registration ablations ----------
\subsubsection{Registration (A14--A15)}

\begin{table}[t]
    \centering
    \small
    \caption{Registration ablations.}
    \label{tab:ablation-registration}
    \begin{tabular}{@{}llp{6.5cm}@{}}
        \toprule
        \textbf{ID} & \textbf{Variable} & \textbf{Variants} \\
        \midrule
        A14 & Registration method & None, Pairwise affine (baseline), MST hierarchical, Congealing \\
        A15 & Impact on clustering & Full pipeline $\pm$ registration; measure ARI improvement \\
        \bottomrule
    \end{tabular}
\end{table}


% ============================================================================
\subsection{Comparative Experiments}
\label{sec:comparisons}

% ---------- Baselines ----------
\subsubsection{Baselines}

\begin{table}[t]
    \centering
    \small
    \caption{Baseline methods for comparison.}
    \label{tab:baselines}
    \begin{tabular}{@{}lp{9cm}@{}}
        \toprule
        \textbf{Baseline} & \textbf{Description} \\
        \midrule
        B1: K-means on HOG & HOG features + $k$-means ($K$ = ground-truth classes). Tests whether the \textit{a contrario} framework adds value beyond simple feature clustering. \\
        B2: Learned + $k$-means & Contrastive ResNet-18 embeddings + $k$-means. Tests the feature representation contribution. \\
        B3: OCR-only & Cluster by OCR-predicted label. Upper bound on linguistic signal; lower bound on what visual clustering should beat for rare/damaged characters. \\
        B4: End-to-end & Fine-tuned segmentation + classification model (if available). \\
        \bottomrule
    \end{tabular}
\end{table}

% ---------- Cross-corpus ----------
\subsubsection{Cross-Corpus Generalisation}

Run the pipeline with \emph{identical hyperparameters} on $\ge 3$ corpora from different periods, scripts, and degradation levels:
\begin{enumerate}[nosep]
    \item Medieval Latin manuscripts (e.g., ICDAR 2017 cBAD).
    \item Early modern printed text (e.g., Google 1000 Books or similar).
    \item Non-Latin script (e.g., Hebrew Dead Sea Scrolls, Arabic papyri).
\end{enumerate}
Report performance drop relative to per-corpus tuning to test the claim that the \textit{a contrario} framework adapts to data distribution.

% ---------- Scalability ----------
\subsubsection{Scalability}

Run the pipeline on subsets of increasing size ($n \in \{500, 1000, 2000, 5000, 10000, 15000\}$) and report wall-clock time and memory per stage.
Verify empirically that complexity bounds hold:
\begin{itemize}[nosep]
    \item NFA computation: $O(N^2)$
    \item HDBSCAN: $O(N \cdot k + N \log N)$
    \item MST alignment: $O(mk \cdot HW)$ per cluster
    \item MRF-BP: $O(|\mathcal{E}| \cdot K \cdot T)$
\end{itemize}


% ============================================================================
\subsection{Figure Plan}
\label{sec:figure-plan}

Table~\ref{tab:figure-plan} summarises the figures planned for the final manuscript.

\begin{table}[t]
    \centering
    \small
    \caption{Planned figures and their contents.}
    \label{tab:figure-plan}
    \begin{tabular}{@{}clp{7.5cm}@{}}
        \toprule
        \textbf{Fig.} & \textbf{Type} & \textbf{Content} \\
        \midrule
        1 & Pipeline & Full pipeline flowchart (extraction $\to$ clustering $\to$ refinement) \\
        2 & Method & \textit{A contrario} matching visualisation: HOG grids, CEMD heatmap, null distribution with NFA threshold \\
        3 & Method & NFA graph structure: nodes at page coordinates, edges coloured by cluster \\
        4 & Method & HDBSCAN condensed tree with stability annotations \\
        5 & Method & Registration comparison: unaligned, Fr\'echet mean, MST, congealing \\
        6 & Method & Noise point gallery (damaged glyphs, artefacts) \\
        \midrule
        7 & Results & Cluster gallery: top-20 character classes + rare classes \\
        8 & Results & Confusion examples: splits, merges, near-misses with NFA values \\
        9 & Results & ARI/NMI/V-measure vs.\ hyperparameters (ablation plots) \\
        10 & Results & Per-class F1 bar chart sorted by frequency \\
        11 & Results & Scalability: wall-clock time vs.\ $N$ (log-log, per stage) \\
        \bottomrule
    \end{tabular}
\end{table}

\begin{figure}[t]
    \centering
    \fbox{\parbox{0.92\linewidth}{\centering\vspace{3cm}%
        \textbf{[PLACEHOLDER --- Figure~\ref{fig:acontrario-viz}]} --- A contrario matching visualisation.
        For a matching and non-matching character pair: (a)~crops, (b)~HOG grids, (c)~per-cell CEMD heatmap, (d)~total $D$ vs.\ null $\mathcal{N}(\mu_0, \sigma_0^2)$ with NFA threshold marked.
    \vspace{3cm}}}
    \caption{Visualisation of the \textit{a contrario} matching process.
    The per-cell CEMD values~(c) are summed to produce the total dissimilarity~$D$, which is compared against the null distribution~(d).
    For matching characters (top), $D$ falls far left of the threshold; for non-matching characters (bottom), $D$ exceeds it.}
    \label{fig:acontrario-viz}
\end{figure}

\begin{figure}[t]
    \centering
    \fbox{\parbox{0.92\linewidth}{\centering\vspace{3cm}%
        \textbf{[PLACEHOLDER --- Figure~\ref{fig:nfa-graph}]} --- NFA graph for a representative page.
        Nodes positioned by page coordinates, edges coloured by cluster.
        Left: full graph.  Right: zoomed region.
    \vspace{3cm}}}
    \caption{NFA similarity graph overlaid on a document page.
    Each node represents a detected character positioned at its page coordinate.
    Edges connect NFA-validated pairs; colours indicate cluster membership.
    Dense intra-cluster connections and sparse inter-cluster edges reflect the quality of the \textit{a contrario} thresholding.}
    \label{fig:nfa-graph}
\end{figure}

\begin{figure}[t]
    \centering
    \fbox{\parbox{0.92\linewidth}{\centering\vspace{3cm}%
        \textbf{[PLACEHOLDER --- Figure~\ref{fig:cluster-gallery}]} --- Qualitative cluster gallery.
        For each of the top-20 most frequent character classes: a row of 10 randomly sampled cluster members.
        Include 2--3 rare character clusters.
    \vspace{3cm}}}
    \caption{Cluster gallery for the most frequent and selected rare character classes.
    Each row shows 10 randomly sampled members of a cluster, demonstrating intra-cluster visual coherence.}
    \label{fig:cluster-gallery}
\end{figure}

\begin{figure}[t]
    \centering
    \fbox{\parbox{0.92\linewidth}{\centering\vspace{3cm}%
        \textbf{[PLACEHOLDER --- Figure~\ref{fig:confusion}]} --- Confusion examples.
        (a)~Splits: same character in two clusters.
        (b)~Merges: different characters in one cluster.
        (c)~Near-misses: characters just above/below the NFA threshold.
        Annotated with NLFA values.
    \vspace{3cm}}}
    \caption{Common clustering errors.
    Splits~(a) arise from within-class variation that exceeds the NFA threshold.
    Merges~(b) occur between visually similar but distinct characters.
    Near-misses~(c) illustrate the sensitivity at the decision boundary.}
    \label{fig:confusion}
\end{figure}

\begin{figure}[t]
    \centering
    \fbox{\parbox{0.92\linewidth}{\centering\vspace{3cm}%
        \textbf{[PLACEHOLDER --- Figure~\ref{fig:ablation-plots}]} --- Clustering metrics (ARI, NMI, V-measure) vs.\ key hyperparameters.
        Panels: (a)~NFA threshold $\varepsilon$, (b)~HDBSCAN \texttt{min\_cluster\_size}, (c)~MRF $\beta$, (d)~Leiden $\gamma$.
    \vspace{3cm}}}
    \caption{Ablation study results.
    Each panel varies one hyperparameter while holding others at their defaults.
    The \textit{a contrario} threshold~(a) shows a plateau of stable performance across two orders of magnitude, confirming the framework's robustness.}
    \label{fig:ablation-plots}
\end{figure}

\begin{figure}[t]
    \centering
    \fbox{\parbox{0.92\linewidth}{\centering\vspace{3cm}%
        \textbf{[PLACEHOLDER --- Figure~\ref{fig:per-class-f1}]} --- Per-class F1 score bar chart.
        Characters sorted by frequency (most frequent left).
        Colour-coded by difficulty: green (F1 $\ge 0.9$), orange (F1 $\ge 0.5$), red (F1 $< 0.5$).
    \vspace{3cm}}}
    \caption{Per-class F1 score sorted by character frequency.
    Common characters achieve near-perfect clustering, while rare characters and visually ambiguous pairs (e.g., `c'/`e', `l'/`i') are more challenging.}
    \label{fig:per-class-f1}
\end{figure}

\begin{figure}[t]
    \centering
    \fbox{\parbox{0.92\linewidth}{\centering\vspace{3cm}%
        \textbf{[PLACEHOLDER --- Figure~\ref{fig:scalability}]} --- Scalability plot.
        Log-log wall-clock time vs.\ number of characters.
        Lines for each pipeline stage: detection, feature extraction, NFA computation, clustering, refinement.
    \vspace{3cm}}}
    \caption{Pipeline scalability.
    Wall-clock time (log scale) as a function of the number of characters (log scale), broken down by pipeline stage.
    NFA computation dominates at large $N$ ($O(N^2)$); all other stages scale sub-quadratically.}
    \label{fig:scalability}
\end{figure}


% ============================================================================
\subsection{Implementation Priority}
\label{sec:priority}

Table~\ref{tab:priority} summarises the implementation priority of each component.

\begin{table}[t]
    \centering
    \small
    \caption{Implementation priority of pipeline components.}
    \label{tab:priority}
    \begin{tabular}{@{}clll@{}}
        \toprule
        \textbf{Priority} & \textbf{Component} & \textbf{Effort} & \textbf{Expected Impact} \\
        \midrule
        P1 & Manual evaluation set & Medium & Unlocks all experiments \\
        P2 & HDBSCAN clustering & Low & Drop-in; handles noise; auto-$K$ \\
        P3 & Empirical background model & Low--Med & Tighter null $\to$ better discrimination \\
        P4 & MST hierarchical registration & Medium & Principled alignment \\
        P5 & K-medoids split/merge & Medium & Replaces 3 ad hoc stages \\
        P6 & MRF fusion (NFA + OCR) & Med--High & Unified probabilistic framework \\
        P7 & Congealing & Medium & Best template quality \\
        P8 & Persistent homology pre-filter & Low & Fast topological rejection \\
        P9 & Affinity Propagation & Low & Interpretable exemplars \\
        P10 & OT on spatial-feature tensor & High & Highest research novelty \\
        \bottomrule
    \end{tabular}
\end{table}
