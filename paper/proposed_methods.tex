% ============================================================================
%  Proposed Methods — Refinement, Registration, and Feature Improvements
% ============================================================================
%
%  This file is designed to be \input{} into main.tex.
% ============================================================================

\section{Proposed Methods}
\label{sec:proposed-methods}

This section presents four categories of improvements over the baseline pipeline: alternative clustering strategies that exploit the continuous NFA signal (Section~\ref{sec:proposed-clustering}), a principled refinement framework (Section~\ref{sec:proposed-refinement}), registration methods for intra-cluster alignment (Section~\ref{sec:proposed-registration}), and deformation-invariant feature descriptors (Section~\ref{sec:proposed-features}).


% ============================================================================
\subsection{Alternative Clustering Methods}
\label{sec:proposed-clustering}

The baseline pipeline binarises NLFA values into graph edges and runs community detection, discarding the graded similarity signal.
We propose three alternatives that operate on the continuous NLFA matrix.


% ---------- HDBSCAN ----------
\subsubsection{HDBSCAN with NFA-Derived Distance}
\label{sec:hdbscan}

HDBSCAN~\cite{mcinneshdbscan2017} provides hierarchical density-based clustering that (a)~preserves the continuous NLFA signal, (b)~automatically determines the number of clusters, and (c)~identifies noise points---damaged or ambiguous glyphs that should not be forced into any cluster.

\paragraph{Distance definition.}
Given the NLFA matrix $S$ (where higher values indicate greater similarity), we define the distance:
\begin{equation}
    d(I_i, I_j) = D_{\max} - \tfrac{1}{2}\bigl[\text{NLFA}(I_i, I_j) + \text{NLFA}(I_j, I_i)\bigr],
    \label{eq:nfa-distance}
\end{equation}
where $D_{\max} = 50$ is a sentinel value (corresponding to NFA $= 10^{-50}$).
Uncomputed pairs receive $d = D_{\max}$.

\paragraph{Mutual reachability.}
For neighbourhood parameter $k = \texttt{min\_samples}$, define the core distance $d_{\text{core}}(I_i) = d(I_i, \mathcal{N}_k(I_i))$ and the mutual reachability distance:
\begin{equation}
    d_{\text{mr}}(I_i, I_j) = \max\!\bigl(d_{\text{core}}(I_i),\; d_{\text{core}}(I_j),\; d(I_i, I_j)\bigr).
    \label{eq:mutual-reachability}
\end{equation}
This inflates distances in sparse regions, making the algorithm robust to the highly variable cluster sizes in character data (common characters like `e' have dense neighbourhoods; rare characters like `z' are sparse).

\paragraph{Hierarchy and cluster extraction.}
A minimum spanning tree on the mutual reachability graph is constructed, then edges are removed in order of increasing $d_{\text{mr}}$.
The resulting condensed tree is traversed bottom-up: at each node, the node's stability
\begin{equation}
    \text{Stab}(C) = \sum_{I_i \in C} \bigl(\lambda_{\text{death}}(I_i, C) - \lambda_{\text{birth}}(C)\bigr), \qquad \lambda = 1/d_{\text{mr}},
    \label{eq:stability}
\end{equation}
is compared to the sum of its children's stabilities.
Flat clusters are extracted via Excess of Mass (EOM) selection.

\paragraph{Noise handling.}
Characters assigned label $-1$ (noise) by HDBSCAN can be post-processed: if their distance to the nearest cluster medoid is below a threshold, they are reassigned; otherwise they remain as singletons.
This is preferable to the baseline where every character is forced into a cluster.

\paragraph{Implementation.}
We use the \texttt{hdbscan} Python package with \texttt{metric='precomputed'}, \texttt{min\_cluster\_size} $\in \{3, 5\}$, and \texttt{min\_samples} $= 3$.
For $N = 15\text{k}$ characters, inference takes seconds on a precomputed distance matrix.

\begin{figure}[t]
    \centering
    \fbox{\parbox{0.92\linewidth}{\centering\vspace{3cm}%
        \textbf{[PLACEHOLDER --- Figure~\ref{fig:hdbscan-tree}]} --- HDBSCAN condensed tree for a representative subset (${\sim}500$ characters), with cluster stability values annotated.
        Selected flat clustering highlighted.
    \vspace{3cm}}}
    \caption{HDBSCAN condensed tree showing the hierarchical density-based decomposition of the NFA distance matrix.
    Cluster stability values (Eq.~\ref{eq:stability}) guide the EOM extraction of the flat clustering (coloured branches).
    Grey branches are classified as noise.}
    \label{fig:hdbscan-tree}
\end{figure}


% ---------- Affinity Propagation ----------
\subsubsection{Affinity Propagation on NFA Similarities}
\label{sec:affinity-propagation}

Affinity Propagation (AP)~\cite{frey2007clustering} operates directly on the pairwise NLFA similarity matrix and simultaneously identifies exemplars (cluster centres) and assignments without requiring $K$.
Exemplars are interpretable as prototypical character instances.

\paragraph{Similarity matrix.}
For $i \neq j$, set $s(I_i, I_j) = \frac{1}{2}[\text{NLFA}(I_i, I_j) + \text{NLFA}(I_j, I_i)]$.
The self-similarity (preference) $p = s(I_i, I_i)$ controls the number of clusters: setting $p = \text{median}(s)$ yields a moderate number of clusters.

\paragraph{Message passing.}
Two types of messages are passed iteratively with damping $\lambda \in [0.5, 0.9]$:

\emph{Responsibility} $r(i,k)$: how well-suited $k$ is as exemplar for $i$, relative to other candidates:
\begin{equation}
    r(i,k) \leftarrow s(i,k) - \max_{k' \neq k}\!\bigl\{a(i,k') + s(i,k')\bigr\}.
\end{equation}

\emph{Availability} $a(i,k)$: how appropriate it is for $i$ to choose $k$:
\begin{equation}
    a(i,k) \leftarrow \min\!\Bigl(0,\; r(k,k) + {\textstyle\sum_{i' \notin \{i,k\}}} \max\bigl(0, r(i',k)\bigr)\Bigr).
\end{equation}

Convergence is declared when exemplar assignments are stable for 15~iterations.

\paragraph{When to prefer AP over HDBSCAN.}
AP is preferable when interpretable exemplars are desired and cluster sizes are relatively balanced.
HDBSCAN is preferable when noise handling is critical and cluster sizes are highly variable.


% ---------- MRF ----------
\subsubsection{MRF with Loopy Belief Propagation}
\label{sec:mrf}

A Markov Random Field jointly optimises cluster assignments using both visual similarity (NFA pairwise potentials) and linguistic information (OCR unary potentials)~\cite{yedidia2003understanding}.

\paragraph{Model definition.}
Let $z_i \in \{1, \ldots, K\}$ be the cluster label for character~$I_i$.
The energy is:
\begin{equation}
    E(\mathbf{z}) = \sum_{i=1}^{N} \psi_i(z_i) + \sum_{(i,j) \in \mathcal{E}} \psi_{ij}(z_i, z_j),
    \label{eq:mrf-energy}
\end{equation}
with unary potentials $\psi_i(z_i = c) = -\log P_{\text{OCR}}(I_i = c)$ encoding OCR predictions, and pairwise potentials (generalised Potts):
\begin{equation}
    \psi_{ij}(z_i, z_j) = \begin{cases} -\beta \cdot \text{NLFA}(I_i, I_j) & \text{if } z_i = z_j, \\ 0 & \text{otherwise,} \end{cases}
    \label{eq:potts}
\end{equation}
where $\beta > 0$ controls the relative weight of visual vs.\ linguistic signal.

\paragraph{Inference.}
Max-sum loopy BP on the sparse NFA graph $\mathcal{G}$ with damping $\lambda = 0.5$:
\begin{equation}
    m_{i \to j}(z_j) \leftarrow \max_{z_i}\!\left[\psi_i(z_i) + \psi_{ij}(z_i, z_j) + \sum_{k \in \mathcal{N}(i) \setminus j} m_{k \to i}(z_i)\right].
\end{equation}
The MAP assignment is $z_i^* = \argmax_{z_i} b_i(z_i)$ where $b_i(z_i) \propto \psi_i(z_i) + \sum_k m_{k \to i}(z_i)$.

\paragraph{Why this subsumes the three refinement stages.}
The unary potentials directly encode the OCR signal (replacing Stage~2), the pairwise potentials capture visual similarity (replacing Stage~1's splitting), and the global energy minimisation handles singletons (replacing Stage~3's PCA z-score test).
The coupling parameter $\beta$ is set to $1/\text{median}(\text{NLFA})$ as a scale-invariant default.


% ============================================================================
\subsection{Principled Refinement: K-Medoids with Split/Merge}
\label{sec:proposed-refinement}

The three ad hoc refinement stages are replaced by a single principled iterative algorithm.

\paragraph{Initialisation.}
Start from the baseline pipeline's output (connected components or Leiden clusters).
Compute medoids: $m_C = \argmin_{I_i \in C} \sum_{I_j \in C} d(I_i, I_j)$.

\paragraph{Reassignment step.}
For each character $I_i$ in cluster $C$, compare its distance to the own medoid $d(I_i, m_C)$ against the $k$-nearest alternative medoids.
If a better match exists, reassign~$I_i$.

\paragraph{Split step.}
For each cluster $C$ with $|C| > \tau_{\text{split}}$, compute the average intra-cluster distance $\bar{d}_C$.
If $\bar{d}_C > \mu_0 + 2\sigma_0$ (where $\mu_0, \sigma_0$ are estimated from the background model), the cluster is likely impure.
Apply spectral bisection: compute the Fiedler vector (second eigenvector of the normalised graph Laplacian of the NLFA subgraph restricted to $C$) and split at the zero crossing.

\paragraph{Merge step.}
For each pair of clusters $(C_a, C_b)$: if $\text{NLFA}(m_{C_a}, m_{C_b}) > \varepsilon_{\text{merge}}$ (medoids are similar) and the fraction of cross-cluster NFA-validated edges exceeds $30\%$, merge $C_a$ and~$C_b$.

\paragraph{Convergence.}
Iterate Reassign $\to$ Split $\to$ Merge $\to$ Recompute medoids until no changes occur.
Reassignment strictly decreases the intra-cluster distance objective $\mathcal{L} = \sum_C \sum_{I_i \in C} d(I_i, m_C)$.
In practice, 3--5 iterations suffice.


% ============================================================================
\subsection{Registration Methods}
\label{sec:proposed-registration}

Given a cluster $C = \{I_1, \ldots, I_m\}$, we require a common coordinate frame for template computation and visual comparison.

% ---------- MST ----------
\subsubsection{MST-Based Hierarchical Alignment}
\label{sec:mst-alignment}

\paragraph{Algorithm.}
\begin{enumerate}[nosep]
    \item Build an alignment graph using $k$-nearest neighbours from the NLFA matrix ($k = 5$).
    \item Compute pairwise alignment residuals $w(I_i, I_j) = \min_{T \in \mathcal{T}} \mathcal{L}_{\text{Lorentz}}(I_i \circ T, I_j)$ as edge weights.
    \item Build the MST (Kruskal's algorithm).
    \item Root at the cluster medoid $r = \argmin_{I_i \in C} \sum_{I_j \in C} d(I_i, I_j)$ to minimise average path length.
    \item Propagate transformations root-to-leaves (BFS): $T_{\text{child}}^{\text{global}} = T_{\text{parent}}^{\text{global}} \circ T_{\text{child} \to \text{parent}}$.
    \item Compute the canonical template as the pixel-wise \emph{median} of aligned images.
\end{enumerate}

\paragraph{Error analysis.}
If each pairwise alignment has residual $\leq \delta$, the accumulated error at depth $L$ is bounded by $\|T_i^{\text{global}} - T_i^{\text{true}}\| \leq L \cdot \delta + O(L^2 \delta^2)$.
For the MST rooted at the medoid, the maximum depth is $O(\log m)$ on average for well-clustered sets.

\paragraph{Drift mitigation.}
After tree propagation, 1--2 rounds of refinement are applied: each character is re-aligned to the current median template (not to its parent), then the template is recomputed.

\paragraph{Complexity.}
$O(mk \cdot HW)$ per cluster (dominated by pairwise alignments), where $k$ is the neighbourhood size and $HW$ is the image resolution.


% ---------- Congealing ----------
\subsubsection{Congealing with Robust Latent Template}
\label{sec:congealing}

Instead of aligning pairwise, jointly optimise all transformations and a latent template~$\bar{I}$~\cite{learned2004data}:
\begin{equation}
    \min_{\bar{I},\, \{T_i\}} \sum_{i=1}^{m} \mathcal{L}\!\bigl(I_i \circ T_i,\; \bar{I}\bigr) + \lambda \sum_{i=1}^{m} \|T_i - \text{Id}\|_F^2.
    \label{eq:congealing}
\end{equation}

\paragraph{Alternating optimisation.}
\begin{enumerate}[nosep]
    \item \textbf{Fix $\{T_i\}$, update $\bar{I}$:}  $\bar{I}(x) = \text{median}_{i=1}^m (I_i \circ T_i)(x)$.
    The pixel-wise median is the ML estimator under Laplacian noise and is robust to outliers.
    \item \textbf{Fix $\bar{I}$, update each $T_i$:}  Standard IC registration of $I_i$ to $\bar{I}$.
\end{enumerate}
Initialise $\bar{I}^{(0)}$ as the medoid character and all $T_i^{(0)} = \text{Id}$.
Convergence typically in 5--10 alternations.

\paragraph{Advantage over Fr\'echet mean.}
The Fr\'echet mean computes the \emph{mean} image, which is blurred by residual misalignment.
Congealing computes the \emph{median} after iterative alignment, producing sharper templates.

\begin{figure}[t]
    \centering
    \fbox{\parbox{0.92\linewidth}{\centering\vspace{3cm}%
        \textbf{[PLACEHOLDER --- Figure~\ref{fig:registration-compare}]} --- Registration comparison for a cluster of 20 characters:
        (a)~unaligned overlay, (b)~Fr\'echet mean (blurry), (c)~MST alignment (sharper), (d)~congealing result.
    \vspace{3cm}}}
    \caption{Comparison of registration methods on a single cluster.
    The MST-based alignment~(c) avoids the blurring of the Fr\'echet mean~(b) by propagating transformations along minimal-residual paths.
    Congealing~(d) further sharpens the template through iterative joint optimisation.}
    \label{fig:registration-compare}
\end{figure}


% ============================================================================
\subsection{Deformation-Invariant Features}
\label{sec:proposed-features}

% ---------- Shape Context ----------
\subsubsection{Shape Context Descriptors}
\label{sec:shape-context}

For each character, $N_s = 100$ points are sampled uniformly from the contour.
For each sample point~$p_i$, the distribution of relative vectors to all other points in log-polar coordinates $(r, \theta)$ is computed~\cite{belongie2002shape}:
\begin{equation}
    h_i(k, l) = \#\{p_j \neq p_i : (r_{ij}, \theta_{ij}) \in \text{bin}(k, l)\},
\end{equation}
with 12 angular bins and 5 radial bins (log-spaced), yielding a 60-dimensional histogram per sample point.
The distance between two characters is the minimum-cost matching via the Hungarian algorithm:
\begin{equation}
    d_{\text{SC}}(I_a, I_b) = \min_{\pi} \frac{1}{N_s}\sum_{i=1}^{N_s} C\!\bigl(h_i^a, h_{\pi(i)}^b\bigr),
\end{equation}
where $C$ is the chi-squared distance.
Shape Context is inherently invariant to translation and optionally to scale (by normalising radial distances).

% ---------- Persistent Homology ----------
\subsubsection{Persistent Homology}
\label{sec:persistent-homology}

The sublevel set filtration of the distance transform $d_I(x) = \min_{y \in \text{ink}} \|x - y\|$ tracks topological features across scales~\cite{edelsbrunner2008persistent}:
\begin{itemize}[nosep]
    \item $H_0$ (connected components): birth/death of connected ink regions.
    \item $H_1$ (loops): birth/death of enclosed holes (e.g., interior of `o', `b', `8').
\end{itemize}
The output is a persistence diagram $\text{Dgm}(I) = \{(b_j, d_j)\}$.
The Wasserstein distance between diagrams provides a stable metric:
\begin{equation}
    W_q\!\bigl(\text{Dgm}(I_a), \text{Dgm}(I_b)\bigr) = \left(\inf_{\gamma} \sum_j \|(b_j, d_j) - \gamma(b_j, d_j)\|_\infty^q\right)^{1/q}\!.
\end{equation}

\paragraph{Topological pre-filter.}
Persistent homology is used as a \emph{fast rejection} step: topologically incompatible pairs (different number of significant loops, e.g., `o' vs.\ `c') are rejected before the expensive NFA comparison.
This reduces the number of pairwise comparisons without loss of recall for topologically compatible characters.
The topological signature (number of significant $H_0$ and $H_1$ features) is computed in $O(HW)$ per character via a single distance transform.

\begin{figure}[t]
    \centering
    \fbox{\parbox{0.92\linewidth}{\centering\vspace{3cm}%
        \textbf{[PLACEHOLDER --- Figure~\ref{fig:noise-gallery}]} --- Gallery of characters classified as noise by HDBSCAN: damaged glyphs, ambiguous strokes, detection artefacts.
    \vspace{3cm}}}
    \caption{Noise points identified by HDBSCAN.
    These characters have low density in the NFA distance space and are correctly excluded from clustering.
    They correspond to damaged glyphs, detection artefacts, and ambiguous stroke fragments that would degrade cluster purity if forced into a cluster.}
    \label{fig:noise-gallery}
\end{figure}
