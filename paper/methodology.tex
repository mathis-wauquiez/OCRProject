% ============================================================================
%  Methodology Section — Historical Document Character Clustering Pipeline
% ============================================================================

\section{Methodology}
\label{sec:methodology}

Our pipeline processes scanned historical document images to produce a vectorised character glossary through four stages: (1)~character extraction, (2)~preprocessing including vectorisation, OCR, and feature computation, (3)~\textit{a contrario} matching and graph-based clustering, and (4)~post-clustering refinement and glossary construction.
Figure~\ref{fig:pipeline-overview} presents an overview.

% ============================================================================
%  Pipeline Overview Figure — full-page flowchart
% ============================================================================
\begin{figure}[t]
    \centering
    \resizebox{0.65\linewidth}{!}{%
    \begin{tikzpicture}[
        >=stealth,
        node distance=0.4cm and 1.0cm,
        stage/.style={
            rectangle, rounded corners=4pt, draw=black!70, thick,
            fill=blue!8, minimum width=3.8cm, minimum height=0.9cm,
            text centered, font=\small\bfseries
        },
        substage/.style={
            rectangle, rounded corners=2pt, draw=black!50,
            fill=white, minimum width=3.4cm, minimum height=0.55cm,
            text centered, font=\footnotesize
        },
        io/.style={
            trapezium, trapezium left angle=70, trapezium right angle=110,
            draw=black!60, fill=orange!10, thick,
            minimum width=2.8cm, minimum height=0.7cm,
            text centered, font=\small
        },
        arrow/.style={->, thick, black!70},
        darrow/.style={->, thick, black!70, dashed},
        group/.style={
            draw=gray!50, dashed, rounded corners=6pt, inner sep=8pt
        },
    ]

    % ── Input ──
    \node[io] (input) {Scanned page images};

    % ── Stage 1: Extraction ──
    \node[stage, below=0.5cm of input] (extraction) {Stage 1: Extraction};
    \node[substage, below=0.1cm of extraction.south west, anchor=north west, xshift=0.2cm]
        (craft) {CRAFT detection};
    \node[substage, below=0.06cm of craft]
        (watershed) {Dual-threshold watershed};
    \node[substage, below=0.06cm of watershed]
        (binarise) {Binarisation + association};
    \node[substage, below=0.06cm of binarise]
        (charfilter) {Character filtering};

    % ── Stage 2: Preprocessing ──
    \node[stage, below=0.6cm of charfilter] (preproc) {Stage 2: Preprocessing};
    \node[substage, below=0.1cm of preproc.south west, anchor=north west, xshift=0.2cm]
        (layout) {Layout analysis};
    \node[substage, below=0.06cm of layout]
        (skew) {Skew estimation};
    \node[substage, below=0.06cm of skew]
        (inkfilt) {Ink filter + vectorisation};
    \node[substage, below=0.06cm of inkfilt]
        (chatocr) {CHAT OCR};
    \node[substage, below=0.06cm of chatocr]
        (hogcomp) {HOG computation};

    % ── Stage 3: Clustering ──
    \node[stage, below=0.6cm of hogcomp] (cluster) {Stage 3: Clustering};
    \node[substage, below=0.1cm of cluster.south west, anchor=north west, xshift=0.2cm]
        (acontrario) {\textit{A contrario} matching};
    \node[substage, below=0.06cm of acontrario]
        (graph) {Graph construction};
    \node[substage, below=0.06cm of graph]
        (leiden) {Leiden community detection};

    % ── Stage 4: Refinement ──
    \node[stage, below=0.6cm of leiden] (refine) {Stage 4: Refinement};
    \node[substage, below=0.1cm of refine.south west, anchor=north west, xshift=0.2cm]
        (hausdorff) {Hausdorff splitting};
    \node[substage, below=0.06cm of hausdorff]
        (ocrrematch) {OCR rematching};
    \node[substage, below=0.06cm of ocrrematch]
        (pcarematch) {PCA z-score rematching};

    % ── Output ──
    \node[io, below=0.5cm of pcarematch] (output) {Vectorised character glossary};

    % ── Arrows ──
    \draw[arrow] (input) -- (extraction);
    \draw[arrow] (charfilter.south) -- ++(0,-0.2) -| (preproc);
    \draw[arrow] (hogcomp.south) -- ++(0,-0.2) -| (cluster);
    \draw[arrow] (leiden.south) -- ++(0,-0.2) -| (refine);
    \draw[arrow] (pcarematch.south) -- ++(0,-0.2) -| (output);

    % ── Side annotations ──
    \node[right=1.0cm of extraction, text=gray, font=\footnotesize\itshape, text width=3cm]
        {Per page\\GPU (CRAFT)};
    \node[right=1.0cm of preproc, text=gray, font=\footnotesize\itshape, text width=3cm]
        {Per page\\CPU + GPU};
    \node[right=1.0cm of cluster, text=gray, font=\footnotesize\itshape, text width=3cm]
        {All pages\\GPU (matching)};
    \node[right=1.0cm of refine, text=gray, font=\footnotesize\itshape, text width=3cm]
        {Per cluster\\GPU (registration)};

    \end{tikzpicture}%
    }
    \caption{Overview of the reverse typography pipeline.
    Scanned page images are processed through four sequential stages to produce a vectorised character glossary.
    Stages are annotated with their computational scope (per-page vs.\ global) and primary hardware usage.}
    \label{fig:pipeline-overview}
\end{figure}



% ============================================================================
\subsection{Character Extraction}
\label{sec:extraction}

We use CRAFT~\cite{baek2019character} (\textit{Character Region Awareness for Text Detection}), a fully convolutional network producing per-pixel text score maps $S_\text{text}(x, y) \in [0,1]$.
We use pre-trained weights without fine-tuning, with magnification ratio~$5.0$ (canvas size 1280\,px).

Connected components are extracted via dual-threshold watershed: pixels with $S_\text{text} \ge \tau_\text{text} = 0.6$ serve as seeds, and basins expand to a lower mask threshold $\tau_\text{mask} = 0.3$.
Nearby components (centroids $< 8$\,px apart) are merged to reconnect split radicals of composite characters.

In parallel, the page is binarised (Otsu's method) and binary connected components are associated with CRAFT detections via negative Mahalanobis distance using the CRAFT region's inertia tensor.
Filters discard components outside the expected size range ($30 \times 30$ to $150 \times 150$\,px), with extreme aspect ratios ($>3$), or insufficient area ($<200$\,px$^2$).


% ============================================================================
\subsection{Preprocessing}
\label{sec:preprocessing}

Each extracted character undergoes preprocessing to produce a vectorised representation, an OCR label, and a HOG descriptor.

\paragraph{Layout analysis and reading order.}
A projection-based layout module identifies columns and rows.
The reading order follows right-to-left, top-to-bottom (classical Chinese), assigning each character an integer index.

\paragraph{Vectorisation via affine scale-space.}
Binary patches are cleaned by morphological filtering (removing small holes and noise specks), then vectorised using affine scale-space smoothing~\cite{alvarez1993axioms,ciomaga2017curvature}.
This geometric PDE evolves each level line of the image according to its curvature:
\begin{equation}
    \frac{\partial u}{\partial t} = |\nabla u|\, \kappa^{1/3},
    \label{eq:gass}
\end{equation}
where $\kappa$ denotes the curvature of the level line through each point.
The $1/3$ exponent makes this evolution affine-invariant: the smoothed shape is independent of the viewing angle, which is essential for comparing characters printed under different conditions.
The resulting smooth level lines are converted to SVG vector outlines.
Document skew is estimated using LSD~\cite{von2010lsd} and corrected via the length-weighted circular mean of line segment orientations:
\begin{equation}
    \theta = \tfrac{1}{2}\arg\!\left(\textstyle\sum_i w_i \, e^{2\mathrm{i}\alpha_i}\right).
    \label{eq:skew}
\end{equation}

\paragraph{Character recognition (CHAT).}
Characters are labelled using a Kraken-based~\cite{kiessling2019kraken} model operating at the subcolumn level.
Predicted character sequences are spatially matched to CRAFT detections via the Hungarian algorithm on vertical distances.
Each character receives a label $\ell_i \in \mathcal{A} \cup \{\square\}$ and confidence $c_i \in [0,1]$.


% ---------- HOG ----------
\paragraph{HOG descriptors.}
Each character is described by a HOG descriptor~\cite{dalal2005histograms} computed on a rendered SVG image ($24 \times 24$\,px cells, 256\,DPI).
Gradients use Gaussian derivative filters ($\sigma = 5$, kernel size~31\,px).
Unsigned orientations are accumulated into $B = 16$ bins per cell via trilinear interpolation:
\begin{equation}
    h_b = \sum_{(x,y) \in \text{cell}} m(x,y) \cdot \bigl[(1 - \alpha)\,\delta_{b, \lfloor \hat\theta \rfloor} + \alpha\,\delta_{b, (\lfloor \hat\theta \rfloor + 1) \bmod B}\bigr],
\end{equation}
where $\hat\theta = \theta \cdot B$ and $\alpha = \hat\theta - \lfloor\hat\theta\rfloor$.
The descriptor is normalised via L2-clip-L2 with clipping threshold $t = 0.2$.

% ============================================================================
%  HOG Descriptor Computation Figure
% ============================================================================
\begin{figure}[t]
    \centering
    \resizebox{\linewidth}{!}{%
    \begin{tikzpicture}[
        >=stealth,
        node distance=0.5cm and 0.8cm,
        box/.style={
            rectangle, rounded corners=2pt, draw=black!60,
            fill=#1, minimum width=2.2cm, minimum height=1.8cm,
            text centered, font=\footnotesize, text width=2cm
        },
        box/.default=white,
        smallbox/.style={
            rectangle, draw=black!40, fill=#1,
            minimum width=1.5cm, minimum height=1.5cm,
            text centered, font=\scriptsize, text width=1.4cm
        },
        smallbox/.default=white,
        arrow/.style={->, thick, black!60},
        label/.style={font=\scriptsize\itshape, text=gray!70},
    ]

    % ── SVG input ──
    \node[box=green!8] (svg) {SVG\\character};

    % ── Render ──
    \node[box=yellow!10, right=0.7cm of svg] (render) {Render\\$24n \times 24n$\,px};

    % ── Gaussian derivative ──
    \node[box=blue!8, right=0.7cm of render] (gradient) {Gaussian\\derivatives\\$\sigma{=}5$};

    % ── Magnitude + Orientation ──
    \node[smallbox=red!8, above right=0.2cm and 0.7cm of gradient] (magn)
        {Magnitude\\$m(x,y)$};
    \node[smallbox=purple!8, below right=0.2cm and 0.7cm of gradient] (orient)
        {Orientation\\$\theta(x,y)$\\(unsigned)};

    % ── Cell decomposition ──
    \node[box=orange!10, right=1.2cm of gradient, yshift=0cm] (cells)
        {Cell grid\\$24{\times}24$\,px};

    % ── Histogram ──
    \node[box=cyan!10, right=0.7cm of cells] (hist)
        {Trilinear\\binning\\$B{=}16$ bins};

    % ── Normalize ──
    \node[box=green!10, right=0.7cm of hist] (norm)
        {L2-clip-L2\\$t{=}0.2$};

    % ── Output ──
    \node[box=gray!10, right=0.7cm of norm] (output)
        {HOG\\descriptor\\$\mathbf{h}^* \in \mathbb{R}^{K \times B}$};

    % ── Arrows ──
    \draw[arrow] (svg) -- (render);
    \draw[arrow] (render) -- (gradient);
    \draw[arrow] (gradient) -- ++(0.5, 0) |- (magn);
    \draw[arrow] (gradient) -- ++(0.5, 0) |- (orient);
    \draw[arrow] (magn) -| (cells);
    \draw[arrow] (orient) -| (cells);
    \draw[arrow] (cells) -- (hist);
    \draw[arrow] (hist) -- (norm);
    \draw[arrow] (norm) -- (output);

    % ── Labels below ──
    \node[label, below=0.15cm of svg] {Input};
    \node[label, below=0.15cm of render] {256\,DPI};
    \node[label, below=0.15cm of gradient] {$k{=}31$\,px};
    \node[label, below=0.15cm of output] {Feature vector};

    \end{tikzpicture}%
    }
    \caption{HOG descriptor computation pipeline.
    Each character's SVG is rendered onto a pixel grid, smoothed with Gaussian derivative filters, decomposed into cells, and each cell's gradient histogram is computed via trilinear interpolation.
    The concatenated histograms are normalised using L2-clip-L2 to produce the final descriptor.}
    \label{fig:hog-pipeline}
\end{figure}



% ============================================================================
\subsection{A Contrario Feature Matching}
\label{sec:acontrario}

Character similarity is assessed using an \textit{a contrario} framework~\cite{desolneux2000meaningful} that provides a statistically grounded matching threshold.

\paragraph{Dissimilarity metric.}
Given two HOG descriptors $\mathbf{h}^A, \mathbf{h}^B$ with $K$ cells of $B$ bins each, we compute the Circular Earth Mover's Distance (CEMD) per cell.
For cumulative difference $X_j = \sum_{i=1}^{j} h^A_{k,i} - \sum_{i=1}^{j} h^B_{k,i}$:
\begin{equation}
    \text{CEMD}(h^A_k, h^B_k) = \min_{s \in \{0,\ldots,B{-}2\}} \frac{1}{B}\sum_{j=1}^{B} |X_j - X_s|.
    \label{eq:cemd}
\end{equation}
The total dissimilarity is $D(\mathbf{h}^A, \mathbf{h}^B) = \sum_{k=1}^{K} \text{CEMD}(h^A_k, h^B_k)$.

\paragraph{Number of False Alarms.}
Under the \textit{a contrario} null model, $D$ is approximately Gaussian with character-specific moments $(\mu_A, \sigma_A^2)$ estimated from all pairwise comparisons.
The NFA is:
\begin{equation}
    \text{NFA}(A,B) = N^2 \cdot \Phi\!\left(\frac{D(\mathbf{h}^A, \mathbf{h}^B) - \mu_A}{\sigma_A}\right),
    \label{eq:nfa}
\end{equation}
where $\Phi$ is the standard normal CDF.
Two characters are meaningfully similar when $\text{NFA} \le \varepsilon = 5 \times 10^{-4}$.
We store the negative log-NFA: $\text{NLFA}(A,B) = -\log\Phi\!\bigl(\frac{D - \mu_A}{\sigma_A}\bigr)$.


% ============================================================================
\subsection{Graph Construction and Community Detection}
\label{sec:graph-clustering}

A similarity graph $G = (V, E, w)$ is built with one vertex per character.
An edge $(i,j)$ is created when both NLFA values exceed threshold $\tau_\text{NFA} = -\log\varepsilon + 2\log N$:
\begin{equation}
    (i,j) \in E \iff \text{NLFA}(i,j) \ge \tau_\text{NFA} \;\wedge\; \text{NLFA}(j,i) \ge \tau_\text{NFA}.
    \label{eq:reciprocal}
\end{equation}
The reciprocal condition discards asymmetric similarities.
Edge weights are the symmetrised NLFA: $w_{ij} = \frac{1}{2}[\text{NLFA}(i,j) + \text{NLFA}(j,i)]$.

Communities are detected using the Leiden algorithm~\cite{traag2019louvain} with the Reichardt--Bornholdt quality function:
\begin{equation}
    \mathcal{Q}_\gamma = \sum_{c} \left[e_c - \gamma \binom{n_c}{2}\right],
\end{equation}
where $e_c$ is the total edge weight within community $c$, $n_c$ its size, and $\gamma = 1.0$.

% ============================================================================
%  Clustering Pipeline Figure — a contrario matching → graph → Leiden
% ============================================================================
\begin{figure}[t]
    \centering
    \resizebox{\linewidth}{!}{%
    \begin{tikzpicture}[
        >=stealth,
        node distance=0.4cm and 1.0cm,
        process/.style={
            rectangle, rounded corners=3pt, draw=black!60, thick,
            fill=#1, minimum width=3.2cm, minimum height=1.0cm,
            text centered, font=\small, text width=3cm
        },
        process/.default=blue!8,
        data/.style={
            rectangle, draw=black!40, fill=gray!5,
            minimum width=2.6cm, minimum height=0.7cm,
            text centered, font=\footnotesize, text width=2.5cm
        },
        arrow/.style={->, thick, black!60},
        note/.style={font=\scriptsize\itshape, text=gray!70, text width=2.8cm},
    ]

    % ── Row 1: HOG features ──
    \node[data] (hog) {HOG descriptors\\$\{\mathbf{h}_i\}_{i=1}^N$};

    % ── Row 2: Dissimilarity ──
    \node[process=yellow!12, below=0.6cm of hog] (dissim)
        {CEMD dissimilarity\\$D(\mathbf{h}^A, \mathbf{h}^B)$};

    % ── Row 3: NFA ──
    \node[process=orange!12, below=0.6cm of dissim] (nfa)
        {NFA computation\\$\text{NLFA}(A,B)$};
    \node[note, right=0.8cm of nfa] {$\mu_A, \sigma_A$ from\\background model};

    % ── Row 4: Thresholding ──
    \node[process=red!10, below=0.6cm of nfa] (threshold)
        {Reciprocal thresholding\\$\varepsilon = 5{\times}10^{-4}$};

    % ── Row 5: Graph ──
    \node[process=green!12, below=0.6cm of threshold] (graph)
        {Similarity graph $G$\\$w_{ij} = \frac{1}{2}(\text{NLFA}_{ij} + \text{NLFA}_{ji})$};

    % ── Row 6: Leiden ──
    \node[process=blue!15, below=0.6cm of graph] (leiden)
        {Leiden algorithm\\$\gamma = 1.0$};

    % ── Row 7: Output ──
    \node[data, below=0.6cm of leiden] (partition)
        {Initial partition\\$\{C_1, \ldots, C_K\}$};

    % ── Arrows ──
    \draw[arrow] (hog) -- (dissim);
    \draw[arrow] (dissim) -- (nfa);
    \draw[arrow] (nfa) -- (threshold);
    \draw[arrow] (threshold) -- (graph);
    \draw[arrow] (graph) -- (leiden);
    \draw[arrow] (leiden) -- (partition);

    % ── Side annotations ──
    \node[note, left=0.8cm of dissim] {Per-cell circular\\EMD (Eq.~\ref{eq:cemd})};
    \node[note, left=0.8cm of threshold] {Both $\text{NLFA}(i,j)$\\and $\text{NLFA}(j,i)$\\must exceed $\tau$};
    \node[note, left=0.8cm of leiden] {RBConfiguration\\vertex partition};

    \end{tikzpicture}%
    }
    \caption{Clustering pipeline: from HOG features to an initial partition.
    Pairwise CEMD dissimilarities are converted into NFA scores under the \textit{a contrario} model, thresholded reciprocally to build a similarity graph, and partitioned by the Leiden algorithm.}
    \label{fig:clustering-pipeline}
\end{figure}



% ============================================================================
\subsection{Cluster Refinement}
\label{sec:refinement}

The Leiden partition may contain impure or over-fragmented clusters.
A sequential three-stage refinement addresses these issues (Fig.~\ref{fig:refinement-pipeline}).

\paragraph{Stage~1: Hausdorff-based splitting.}
Large clusters ($\ge 5$ members) are inspected using pairwise Hausdorff distances~\cite{rony2025hausdorff} on registered binary images.
Registration uses multiscale inverse compositional (IC) alignment~\cite{briand2018ipol} with a Lorentzian robust error.
The Hausdorff distance matrix is used to build an average-linkage dendrogram, cut at $\tau_\text{split} = 21.5$.

\paragraph{Stage~2: OCR-based rematching.}
Small clusters ($\le 3$ members) are merged into the largest cluster sharing the same dominant OCR label.
This fast, label-driven step handles fragmented Leiden partitions.

\paragraph{Stage~3: PCA z-score rematching.}
Remaining small clusters are tested against $k=5$ nearest large clusters ($\ge 10$ members) via image-space PCA compatibility.
The query is registered against each candidate's most central member, projected into the candidate's PCA space ($d=5$ components), and merged if $\max_j |z_j| < 3.0$.

% ============================================================================
%  Refinement Pipeline Figure — three-stage post-partition refinement
% ============================================================================
\begin{figure}[t]
    \centering
    \resizebox{\linewidth}{!}{%
    \begin{tikzpicture}[
        >=stealth,
        node distance=0.3cm and 1.5cm,
        stage/.style={
            rectangle, rounded corners=4pt, draw=#1!70, thick,
            fill=#1!10, minimum width=4.5cm, minimum height=1.2cm,
            text centered, font=\small\bfseries, text width=4.2cm
        },
        detail/.style={
            rectangle, rounded corners=2pt, draw=black!30,
            fill=white, minimum width=4.0cm, minimum height=0.6cm,
            text centered, font=\scriptsize, text width=3.8cm
        },
        io/.style={
            rounded rectangle, draw=black!50, fill=gray!8,
            minimum width=3.5cm, minimum height=0.8cm,
            text centered, font=\small
        },
        arrow/.style={->, very thick, black!60},
        note/.style={font=\scriptsize\itshape, text=gray!60, text width=3.2cm, align=left},
    ]

    % ── Input ──
    \node[io] (input) {Leiden partition};

    % ── Stage 1: Hausdorff split ──
    \node[stage=red, below=0.7cm of input] (split) {Hausdorff Splitting};
    \node[detail, below=0.1cm of split] (split1) {Pairwise IC registration + Hausdorff};
    \node[detail, below=0.05cm of split1] (split2) {Average linkage dendrogram};
    \node[detail, below=0.05cm of split2] (split3) {Cut at $\tau_\text{split}$; unknowns $\to$ largest sub-cluster};

    \node[note, right=0.8cm of split] {Impure clusters\\$\to$ visually coherent\\sub-clusters};

    % ── Stage 2: OCR rematch ──
    \node[stage=blue, below=0.7cm of split3] (ocr) {OCR Rematching};
    \node[detail, below=0.1cm of ocr] (ocr1) {Small clusters ($\le 3$): find dominant label};
    \node[detail, below=0.05cm of ocr1] (ocr2) {Merge into largest cluster with same label};

    \node[note, right=0.8cm of ocr] {Fast, label-driven\\(no GPU)};

    % ── Stage 3: PCA rematch ──
    \node[stage=green!70!black, below=0.7cm of ocr2] (pca) {PCA Z-Score Rematching};
    \node[detail, below=0.1cm of pca] (pca1) {$k$-NN candidate clusters (HOG distance)};
    \node[detail, below=0.05cm of pca1] (pca2) {IC register query $\to$ anchor};
    \node[detail, below=0.05cm of pca2] (pca3) {PCA projection + z-score test};

    \node[note, right=0.8cm of pca] {Image-space\\compatibility test\\for remaining singletons};

    % ── Output ──
    \node[io, below=0.7cm of pca3] (output) {Refined clusters};

    % ── Arrows ──
    \draw[arrow] (input) -- (split);
    \draw[arrow] (split3.south) -- ++(0,-0.2) -| ([xshift=0cm]ocr.north);
    \draw[arrow] (ocr2.south) -- ++(0,-0.2) -| ([xshift=0cm]pca.north);
    \draw[arrow] (pca3.south) -- ++(0,-0.2) -| ([xshift=0cm]output.north);

    % ── Stage labels ──
    \node[left=0.6cm of split, font=\footnotesize\bfseries, text=red!70] {Stage 1};
    \node[left=0.6cm of ocr, font=\footnotesize\bfseries, text=blue!70] {Stage 2};
    \node[left=0.6cm of pca, font=\footnotesize\bfseries, text=green!50!black] {Stage 3};

    \end{tikzpicture}%
    }
    \caption{Three-stage cluster refinement pipeline.
    After the initial Leiden partition, clusters are sequentially split by visual dissimilarity (Stage~1), merged by OCR label (Stage~2), and merged by image-space PCA compatibility (Stage~3).
    Each stage is composable and produces structured diagnostics for reporting.}
    \label{fig:refinement-pipeline}
\end{figure}



% ============================================================================
\subsection{Evaluation Metrics}
\label{sec:metrics}

Clustering quality is evaluated using: Adjusted Rand Index (ARI)~\cite{hubert1985comparing} as the primary metric, Normalised Mutual Information (NMI), homogeneity/completeness/V-measure~\cite{rosenberg2007v}, pairwise F1 score, and Hungarian-matching accuracy.
Characters with unknown ground-truth labels are excluded from all metric computations.


% ============================================================================
\subsection{Glossary Construction}
\label{sec:glossary}

The final output of the pipeline is a \emph{character glossary}: for each cluster, we select the most central member (minimising mean intra-cluster dissimilarity) as the representative, and record the dominant OCR label, cluster size, and purity.
The glossary is sorted by frequency, producing an inventory of all distinct character forms in the book along with their vectorised outlines---the desired output of reverse typography.
