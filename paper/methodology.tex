% ============================================================================
%  Methodology Section — Historical Document Character Clustering Pipeline
% ============================================================================
%
%  This file is designed to be \input{} into a main paper document.
%  It assumes the preamble loads:
%    tikz, pgfplots, amsmath, amssymb, booktabs, graphicx, xcolor,
%    algorithm2e or algorithmic, hyperref, subcaption
%
%  Figures marked [PLACEHOLDER] should be replaced with actual data figures.
% ============================================================================

\section{Methodology}
\label{sec:methodology}

Our pipeline processes scanned historical document images to extract, describe, and cluster individual character glyphs.
It consists of four sequential stages: (1)~character extraction from page images, (2)~preprocessing of extracted patches including vectorization, OCR, and feature computation, (3)~\textit{a contrario} feature matching and graph-based clustering, and (4)~post-clustering refinement.
Figure~\ref{fig:pipeline-overview} presents an overview of the full pipeline.

% ============================================================================
%  Pipeline Overview Figure — full-page flowchart
% ============================================================================
\begin{figure}[t]
    \centering
    \resizebox{0.65\linewidth}{!}{%
    \begin{tikzpicture}[
        >=stealth,
        node distance=0.4cm and 1.0cm,
        stage/.style={
            rectangle, rounded corners=4pt, draw=black!70, thick,
            fill=blue!8, minimum width=3.8cm, minimum height=0.9cm,
            text centered, font=\small\bfseries
        },
        substage/.style={
            rectangle, rounded corners=2pt, draw=black!50,
            fill=white, minimum width=3.4cm, minimum height=0.55cm,
            text centered, font=\footnotesize
        },
        io/.style={
            trapezium, trapezium left angle=70, trapezium right angle=110,
            draw=black!60, fill=orange!10, thick,
            minimum width=2.8cm, minimum height=0.7cm,
            text centered, font=\small
        },
        arrow/.style={->, thick, black!70},
        darrow/.style={->, thick, black!70, dashed},
        group/.style={
            draw=gray!50, dashed, rounded corners=6pt, inner sep=8pt
        },
    ]

    % ── Input ──
    \node[io] (input) {Scanned page images};

    % ── Stage 1: Extraction ──
    \node[stage, below=0.5cm of input] (extraction) {Stage 1: Extraction};
    \node[substage, below=0.1cm of extraction.south west, anchor=north west, xshift=0.2cm]
        (craft) {CRAFT detection};
    \node[substage, below=0.06cm of craft]
        (watershed) {Dual-threshold watershed};
    \node[substage, below=0.06cm of watershed]
        (binarise) {Binarisation + association};
    \node[substage, below=0.06cm of binarise]
        (charfilter) {Character filtering};

    % ── Stage 2: Preprocessing ──
    \node[stage, below=0.6cm of charfilter] (preproc) {Stage 2: Preprocessing};
    \node[substage, below=0.1cm of preproc.south west, anchor=north west, xshift=0.2cm]
        (layout) {Layout analysis};
    \node[substage, below=0.06cm of layout]
        (skew) {Skew estimation};
    \node[substage, below=0.06cm of skew]
        (inkfilt) {Ink filter + vectorisation};
    \node[substage, below=0.06cm of inkfilt]
        (chatocr) {CHAT OCR};
    \node[substage, below=0.06cm of chatocr]
        (hogcomp) {HOG computation};

    % ── Stage 3: Clustering ──
    \node[stage, below=0.6cm of hogcomp] (cluster) {Stage 3: Clustering};
    \node[substage, below=0.1cm of cluster.south west, anchor=north west, xshift=0.2cm]
        (acontrario) {\textit{A contrario} matching};
    \node[substage, below=0.06cm of acontrario]
        (graph) {Graph construction};
    \node[substage, below=0.06cm of graph]
        (leiden) {Leiden community detection};

    % ── Stage 4: Refinement ──
    \node[stage, below=0.6cm of leiden] (refine) {Stage 4: Refinement};
    \node[substage, below=0.1cm of refine.south west, anchor=north west, xshift=0.2cm]
        (hausdorff) {Hausdorff splitting};
    \node[substage, below=0.06cm of hausdorff]
        (ocrrematch) {OCR rematching};
    \node[substage, below=0.06cm of ocrrematch]
        (pcarematch) {PCA z-score rematching};

    % ── Output ──
    \node[io, below=0.5cm of pcarematch] (output) {Vectorised character glossary};

    % ── Arrows ──
    \draw[arrow] (input) -- (extraction);
    \draw[arrow] (charfilter.south) -- ++(0,-0.2) -| (preproc);
    \draw[arrow] (hogcomp.south) -- ++(0,-0.2) -| (cluster);
    \draw[arrow] (leiden.south) -- ++(0,-0.2) -| (refine);
    \draw[arrow] (pcarematch.south) -- ++(0,-0.2) -| (output);

    % ── Side annotations ──
    \node[right=1.0cm of extraction, text=gray, font=\footnotesize\itshape, text width=3cm]
        {Per page\\GPU (CRAFT)};
    \node[right=1.0cm of preproc, text=gray, font=\footnotesize\itshape, text width=3cm]
        {Per page\\CPU + GPU};
    \node[right=1.0cm of cluster, text=gray, font=\footnotesize\itshape, text width=3cm]
        {All pages\\GPU (matching)};
    \node[right=1.0cm of refine, text=gray, font=\footnotesize\itshape, text width=3cm]
        {Per cluster\\GPU (registration)};

    \end{tikzpicture}%
    }
    \caption{Overview of the reverse typography pipeline.
    Scanned page images are processed through four sequential stages to produce a vectorised character glossary.
    Stages are annotated with their computational scope (per-page vs.\ global) and primary hardware usage.}
    \label{fig:pipeline-overview}
\end{figure}



% ============================================================================
\subsection{Character Extraction}
\label{sec:extraction}

The extraction stage takes a scanned page image as input and produces a set of bounding boxes, one per detected character.
It combines a learned text-detection network with classical image-processing steps to achieve high recall without requiring character-level annotations.

% ---------- CRAFT ----------
\subsubsection{Text Region Detection with CRAFT}
\label{sec:craft}

We use CRAFT~\cite{baek2019character} (\textit{Character Region Awareness for Text Detection}) as the backbone detector.
CRAFT is a fully convolutional network built on a VGG-16 backbone with batch normalisation, followed by a U-Net-style decoder that produces two per-pixel score maps at half the input resolution:
\begin{itemize}
    \item a \emph{text score map} $S_\text{text}(x, y) \in [0,1]$, indicating the probability that pixel $(x,y)$ belongs to a character centre region;
    \item a \emph{link score map} $S_\text{link}(x, y) \in [0,1]$, indicating affinity between adjacent characters (unused in our current configuration).
\end{itemize}
We use the pre-trained weights from \texttt{craft\_mlt\_25k.pth} without fine-tuning.
The input page is rescaled with a magnification ratio of $5.0$ (canvas size 1280\,px) to ensure small characters are sufficiently resolved.


% ---------- Watershed ----------
\subsubsection{Dual-Threshold Watershed Segmentation}
\label{sec:watershed}

Connected components are extracted from the CRAFT text score map via a dual-threshold watershed procedure:
\begin{enumerate}
    \item \textbf{Seed detection.}  Pixels with $S_\text{text} \ge \tau_\text{text}$ (default $\tau_\text{text} = 0.6$) are used as seeds for the watershed algorithm.
    \item \textbf{Mask expansion.}  The basin mask is defined at a lower threshold $\tau_\text{mask} = 0.3$, allowing character regions to expand into areas with moderate CRAFT scores.  This is particularly important for composite characters (e.g.\ CJK radicals such as~``\begin{CJK}{UTF8}{gbsn}蒸\end{CJK}'') where sub-parts may receive lower text scores.
    \item \textbf{Component merging.}  Nearby components whose centroids are closer than $d_\text{min} = 8$\,px are merged, connecting split radicals of composite characters.
\end{enumerate}

% ---------- Filtering ----------
\subsubsection{Component Filtering and Character Segmentation}
\label{sec:component-filtering}

The raw CRAFT components are filtered by minimum area ($\ge 18$\,px$^2$) and aspect ratio ($0.2 \le \text{AR} \le 5.0$) to remove noise and non-character detections.

In parallel, the page image is binarised using Otsu's method, and connected components are extracted from the binary image.
Line-like artefacts are removed by filtering components with extreme aspect ratios ($> 20$) and major axis lengths ($> 100$\,px).

The binary image components are then associated with CRAFT detections via a spatial proximity metric.
For each CRAFT detection, its centroid is mapped into the CRAFT-preprocessed coordinate space, and the negative Mahalanobis distance to each binary component centroid is computed using the CRAFT region's inertia tensor as the covariance matrix.
Each binary component is assigned to the closest CRAFT detection (if the distance is below a threshold), forming the final character segmentation.

Additional filters discard characters whose bounding boxes fall outside the expected size range ($30 \times 30$ to $150 \times 150$\,px), have extreme aspect ratios ($> 3$), excessive fill ($> 90\%$), or insufficient area ($< 200$\,px$^2$).
Characters whose centroids are too close to the contours of large connected components ($> 4000$\,px area) are also removed, as these typically correspond to noise at the edges of decorative elements.

\begin{figure}[t]
    \centering
    \fbox{\parbox{0.95\linewidth}{\centering\vspace{3cm}%
        \textbf{[PLACEHOLDER]} --- Example showing a page crop with: (a)~CRAFT score map, (b)~dual-threshold watershed, (c)~binary image with component overlay, (d)~final character bounding boxes after filtering.
    \vspace{3cm}}}
    \caption{Character extraction pipeline on an example page region.
    From the raw CRAFT score map~(a), a dual-threshold watershed detects character seeds~(b).
    Binary image components~(c) are spatially associated with CRAFT detections to produce the final character segmentation~(d).}
    \label{fig:extraction-example}
\end{figure}

% ============================================================================
\subsection{Preprocessing}
\label{sec:preprocessing}

Each extracted character undergoes a preprocessing pipeline that produces (i)~a vectorised representation, (ii)~an OCR label, and (iii)~a HOG feature descriptor.
This stage also establishes the reading order of characters within the document.

% ---------- Layout Analysis ----------
\subsubsection{Layout Analysis and Reading Order}
\label{sec:layout}

A layout analysis module identifies the column structure of the page by projecting the binary character mask onto the horizontal axis and detecting runs above an Otsu-derived threshold.
Within each column, rows are identified by a transposed projection.
Subcolumns within each row are detected by a further projection pass, with narrow gaps (smaller than $0.3 \times \text{row height}$ or $3$\,px) merged to avoid false splits from internal character stroke gaps.

The reading order follows the conventional right-to-left, top-to-bottom direction of classical Chinese texts: columns are traversed from right to left, and within each column, characters are read top to bottom.
Each character receives an integer reading-order index used in downstream visualisation and analysis.

% ---------- Skew ----------
\subsubsection{Skew Estimation}
\label{sec:skew}

Document skew is estimated using the Line Segment Detector (LSD)~\cite{von2010lsd} with advanced refinement.
Long line segments ($> 100$\,px) are classified as horizontal or vertical (within $\pm\pi/8$), and the skew angle $\theta$ is computed as the length-weighted circular mean of their orientations:
\begin{equation}
    \theta = \frac{1}{2}\arg\!\left(\sum_i w_i \, e^{2i\alpha_i}\right),
    \label{eq:skew}
\end{equation}
where $w_i$ is the length and $\alpha_i$ the orientation of line segment~$i$.
This angle is used to deskew the vectorised character representations.

% ---------- Ink Filter ----------
\subsubsection{Ink Filtering}
\label{sec:ink-filter}

Before vectorisation, each binary character patch is cleaned by a two-pass morphological filter:
\begin{enumerate}
    \item Small white connected components ($< 5$\,px) are removed (fills small holes in strokes).
    \item Small black connected components ($< 10$\,px) are removed (removes isolated noise specks).
\end{enumerate}

% ---------- Vectorisation ----------
\subsubsection{Vectorisation}
\label{sec:vectorisation}

Cleaned binary patches are converted into Scalable Vector Graphics (SVG) representations using a C++ vectorisation tool based on the Potrace algorithm.
The vectoriser traces merged shape outlines with a smoothing scale of~$1.0$ and accuracy threshold of~$1.0$.
The resulting SVGs encode each character as a set of B\'ezier curves, providing a resolution-independent representation.
After vectorisation, each SVG is deskewed by applying the estimated page rotation (Eq.~\ref{eq:skew}) as a homographic transformation.

The SVG representation serves as the canonical form from which all downstream renderings (for HOG computation, for image registration) are generated.

% ---------- OCR ----------
\subsubsection{Character Recognition with CHAT}
\label{sec:chat-ocr}

Characters are labelled using a Kraken-based~\cite{kiessling2019kraken} recognition model (CHAT), which operates at the subcolumn level rather than on individual characters:

\begin{enumerate}
    \item For each subcolumn identified by the layout analysis (Section~\ref{sec:layout}), the grayscale page image is cropped to the bounding box of its characters and binarised.
    \item A synthetic baseline is constructed as a vertical line at the median $x$-coordinate of the CRAFT centroids, spanning the full crop height.
    This provides the line geometry expected by Kraken without requiring a separate segmentation model.
    \item \texttt{kraken.rpred.rpred()} is called with \texttt{text\_direction="vertical-rl"}, which handles line extraction, dewarping, and character-by-character inference.
    \item The predicted character sequence is spatially matched to CRAFT detections via the Hungarian algorithm applied to a cost matrix of vertical distances between Kraken cut polygon centres and CRAFT centroid $y$-coordinates.
    Matches farther than $1.5 \times$ the average character height are rejected.
    Unmatched CRAFT centroids receive the unknown label~``$\square$''.
\end{enumerate}

Each character thus receives a recognised label $\ell_i \in \mathcal{A} \cup \{\square\}$ and a confidence score $c_i \in [0,1]$, where $\mathcal{A}$ is the model's alphabet and $\square$ denotes an unrecognised character.

\begin{figure}[t]
    \centering
    \fbox{\parbox{0.95\linewidth}{\centering\vspace{3cm}%
        \textbf{[PLACEHOLDER]} --- Example showing a subcolumn with: (left)~binarised crop with CRAFT centroids and Kraken cut boundaries, (right)~predicted characters at matched positions.
    \vspace{3cm}}}
    \caption{CHAT OCR on an example subcolumn.
    Left: the binarised subcolumn crop with CRAFT centroids (red crosses), the synthetic vertical baseline (yellow), and Kraken cut boundaries (cyan dashed lines).
    Right: predicted characters coloured by confidence (green $\ge 0.8$, orange $\ge 0.5$, red $< 0.5$).}
    \label{fig:chat-ocr-example}
\end{figure}


% ---------- HOG ----------
\subsubsection{Histogram of Oriented Gradients}
\label{sec:hog}

Each character is described by a Histogram of Oriented Gradients (HOG)~\cite{dalal2005histograms} computed on a rendered image of its SVG.
The SVG is rendered onto a $24 \times 24$\,px-aligned grid at 256\,DPI with a binarisation threshold of~128, producing a clean grayscale image.

\paragraph{Gradient computation.}
Gradients are computed using Gaussian derivative filters with standard deviation $\sigma = 5$ and kernel size $6\sigma + 1 = 31$\,px.
Horizontal and vertical derivatives are obtained via separable convolution: Gaussian smoothing in one direction, followed by the corresponding Gaussian derivative in the other:
\begin{equation}
    \frac{\partial I}{\partial x} = \left(G_\sigma * I\right) \circledast d_\sigma,
    \qquad
    \frac{\partial I}{\partial y} = \left(d_\sigma * I\right) \circledast G_\sigma,
\end{equation}
where $G_\sigma$ is a 1D Gaussian kernel, $d_\sigma$ its derivative, and $*$, $\circledast$ denote vertical and horizontal convolution respectively.

Gradient magnitudes and \emph{unsigned} orientations are computed as:
\begin{equation}
    m(x,y) = \sqrt{\left(\frac{\partial I}{\partial x}\right)^2 + \left(\frac{\partial I}{\partial y}\right)^2},
    \qquad
    \theta(x,y) = \left(\operatorname{atan2}\!\left(\frac{\partial I}{\partial y}, \frac{\partial I}{\partial x}\right) \bmod \pi\right) / \pi.
\end{equation}
Unsigned gradients are used because character strokes have no inherent directionality: a dark-to-light transition and a light-to-dark transition along the same stroke should produce the same descriptor.

\paragraph{Histogram computation.}
The rendered image is divided into a grid of $24 \times 24$\,px cells.
Within each cell, gradient magnitudes are accumulated into $B = 16$ orientation bins using trilinear interpolation:
\begin{equation}
    h_b = \sum_{(x,y) \in \text{cell}} m(x,y) \cdot \bigl[(1 - \alpha)\,\delta_{b, \lfloor \hat\theta \rfloor} + \alpha\,\delta_{b, (\lfloor \hat\theta \rfloor + 1) \bmod B}\bigr],
\end{equation}
where $\hat\theta(x,y) = \theta(x,y) \cdot B$ and $\alpha = \hat\theta - \lfloor\hat\theta\rfloor$.

\paragraph{Normalisation.}
The descriptor is normalised at the patch level (all cells jointly) using the two-step SIFT-style procedure:
\begin{equation}
    \hat{\mathbf{h}} = \frac{\mathbf{h}}{\|\mathbf{h}\|}, \qquad
    \tilde{\mathbf{h}} = \operatorname{clip}(\hat{\mathbf{h}}, -t, t), \qquad
    \mathbf{h}^* = \frac{\tilde{\mathbf{h}}}{\|\tilde{\mathbf{h}}\|},
\end{equation}
with clipping threshold $t = 0.2$.
This prevents any single dominant gradient direction from dominating the descriptor.

% ============================================================================
%  HOG Descriptor Computation Figure
% ============================================================================
\begin{figure}[t]
    \centering
    \resizebox{\linewidth}{!}{%
    \begin{tikzpicture}[
        >=stealth,
        node distance=0.5cm and 0.8cm,
        box/.style={
            rectangle, rounded corners=2pt, draw=black!60,
            fill=#1, minimum width=2.2cm, minimum height=1.8cm,
            text centered, font=\footnotesize, text width=2cm
        },
        box/.default=white,
        smallbox/.style={
            rectangle, draw=black!40, fill=#1,
            minimum width=1.5cm, minimum height=1.5cm,
            text centered, font=\scriptsize, text width=1.4cm
        },
        smallbox/.default=white,
        arrow/.style={->, thick, black!60},
        label/.style={font=\scriptsize\itshape, text=gray!70},
    ]

    % ── SVG input ──
    \node[box=green!8] (svg) {SVG\\character};

    % ── Render ──
    \node[box=yellow!10, right=0.7cm of svg] (render) {Render\\$24n \times 24n$\,px};

    % ── Gaussian derivative ──
    \node[box=blue!8, right=0.7cm of render] (gradient) {Gaussian\\derivatives\\$\sigma{=}5$};

    % ── Magnitude + Orientation ──
    \node[smallbox=red!8, above right=0.2cm and 0.7cm of gradient] (magn)
        {Magnitude\\$m(x,y)$};
    \node[smallbox=purple!8, below right=0.2cm and 0.7cm of gradient] (orient)
        {Orientation\\$\theta(x,y)$\\(unsigned)};

    % ── Cell decomposition ──
    \node[box=orange!10, right=1.2cm of gradient, yshift=0cm] (cells)
        {Cell grid\\$24{\times}24$\,px};

    % ── Histogram ──
    \node[box=cyan!10, right=0.7cm of cells] (hist)
        {Trilinear\\binning\\$B{=}16$ bins};

    % ── Normalize ──
    \node[box=green!10, right=0.7cm of hist] (norm)
        {L2-clip-L2\\$t{=}0.2$};

    % ── Output ──
    \node[box=gray!10, right=0.7cm of norm] (output)
        {HOG\\descriptor\\$\mathbf{h}^* \in \mathbb{R}^{K \times B}$};

    % ── Arrows ──
    \draw[arrow] (svg) -- (render);
    \draw[arrow] (render) -- (gradient);
    \draw[arrow] (gradient) -- ++(0.5, 0) |- (magn);
    \draw[arrow] (gradient) -- ++(0.5, 0) |- (orient);
    \draw[arrow] (magn) -| (cells);
    \draw[arrow] (orient) -| (cells);
    \draw[arrow] (cells) -- (hist);
    \draw[arrow] (hist) -- (norm);
    \draw[arrow] (norm) -- (output);

    % ── Labels below ──
    \node[label, below=0.15cm of svg] {Input};
    \node[label, below=0.15cm of render] {256\,DPI};
    \node[label, below=0.15cm of gradient] {$k{=}31$\,px};
    \node[label, below=0.15cm of output] {Feature vector};

    \end{tikzpicture}%
    }
    \caption{HOG descriptor computation pipeline.
    Each character's SVG is rendered onto a pixel grid, smoothed with Gaussian derivative filters, decomposed into cells, and each cell's gradient histogram is computed via trilinear interpolation.
    The concatenated histograms are normalised using L2-clip-L2 to produce the final descriptor.}
    \label{fig:hog-pipeline}
\end{figure}



% ============================================================================
\subsection{A Contrario Feature Matching}
\label{sec:acontrario}

Character similarity is assessed using an \textit{a contrario} framework~\cite{desolneux2000meaningful} that provides a principled, statistically grounded threshold for declaring two characters as similar.

% ---------- Dissimilarity ----------
\subsubsection{Dissimilarity Metric: Circular Earth Mover's Distance}
\label{sec:cemd}

Given two HOG descriptors $\mathbf{h}^A = (h^A_1, \ldots, h^A_K)$ and $\mathbf{h}^B = (h^B_1, \ldots, h^B_K)$, where $K$ is the number of cells and each $h^A_k, h^B_k \in \mathbb{R}^B$ is a normalised $B$-bin histogram, we compute a per-cell dissimilarity using the Circular Earth Mover's Distance (CEMD).

For a single cell pair $(h^A_k, h^B_k)$, let $X_j = \sum_{i=1}^{j} h^A_{k,i} - \sum_{i=1}^{j} h^B_{k,i}$ be the cumulative difference.
The CEMD is:
\begin{equation}
    \text{CEMD}(h^A_k, h^B_k) = \min_{s \in \{0,\ldots,B-2\}} \frac{1}{B}\sum_{j=1}^{B} |X_j - X_s|,
    \label{eq:cemd}
\end{equation}
which finds the optimal circular alignment of the two histograms before computing the $L^1$ transport cost.

The total dissimilarity between two characters is the sum over all cells:
\begin{equation}
    D(\mathbf{h}^A, \mathbf{h}^B) = \sum_{k=1}^{K} \text{CEMD}(h^A_k, h^B_k).
    \label{eq:total-dissimilarity}
\end{equation}

% ---------- NLFA ----------
\subsubsection{Number of False Alarms}
\label{sec:nlfa}

Under the \textit{a contrario} model, we assume that for a query character $A$, the cell-level dissimilarities $D_k = \text{CEMD}(h^A_k, h^B_k)$ to a random background character $B$ are independent random variables.
By the central limit theorem, the total dissimilarity $D = \sum_k D_k$ is approximately normally distributed with mean $\mu_A = \sum_k \mathbb{E}[D_k]$ and variance $\sigma_A^2 = \sum_k \operatorname{Var}[D_k]$, where the moments are estimated from all pairwise comparisons.

The Number of False Alarms (NFA) for the pair $(A, B)$ is:
\begin{equation}
    \text{NFA}(A,B) = N^2 \cdot \Phi\!\left(\frac{D(\mathbf{h}^A, \mathbf{h}^B) - \mu_A}{\sigma_A}\right),
    \label{eq:nfa}
\end{equation}
where $\Phi$ is the standard normal CDF and $N$ is the number of characters.
The negative log NFA (NLFA) is stored as:
\begin{equation}
    \text{NLFA}(A,B) = -\log\Phi\!\left(\frac{D - \mu_A}{\sigma_A}\right).
\end{equation}
Two characters are considered meaningfully similar when $\text{NFA}(A,B) \le \varepsilon$, i.e.\ $\text{NLFA}(A,B) \ge -\log\varepsilon + 2\log N$.
The parameter $\varepsilon$ controls the expected number of false detections under the null model; we use $\varepsilon = 5 \times 10^{-4}$.


% ============================================================================
\subsection{Graph Construction and Community Detection}
\label{sec:graph-clustering}

% ---------- Graph ----------
\subsubsection{Graph Construction}
\label{sec:graph-construction}

A similarity graph $G = (V, E, w)$ is built with one vertex per character ($|V| = N$).
An edge $(i, j)$ is created when both NLFA values exceed the threshold:
\begin{equation}
    (i,j) \in E \iff \text{NLFA}(i,j) \ge \tau_\text{NFA} \;\wedge\; \text{NLFA}(j,i) \ge \tau_\text{NFA},
    \label{eq:reciprocal}
\end{equation}
where $\tau_\text{NFA} = -\log\varepsilon + 2\log N$.
The reciprocal condition enforces that both characters consider each other as meaningful matches, discarding asymmetric similarities.
Edge weights are set to the symmetrised NLFA: $w_{ij} = \frac{1}{2}[\text{NLFA}(i,j) + \text{NLFA}(j,i)]$.


% ---------- Leiden ----------
\subsubsection{Community Detection with Leiden}
\label{sec:leiden}

Communities in $G$ are detected using the Leiden algorithm~\cite{traag2019louvain} with the Reichardt--Bornholdt configuration model quality function (RBConfiguration), which optimises:
\begin{equation}
    \mathcal{Q}_\gamma = \sum_{c} \left[e_c - \gamma \binom{n_c}{2}\right],
\end{equation}
where $e_c$ is the total weight of edges within community $c$, $n_c$ its size, and $\gamma$ is the resolution parameter.
Higher $\gamma$ values produce more, smaller clusters.
We use $\gamma = 1.0$ and run until convergence (\texttt{n\_iterations=-1}) with a fixed seed for reproducibility.

% ============================================================================
%  Clustering Pipeline Figure — a contrario matching → graph → Leiden
% ============================================================================
\begin{figure}[t]
    \centering
    \resizebox{\linewidth}{!}{%
    \begin{tikzpicture}[
        >=stealth,
        node distance=0.4cm and 1.0cm,
        process/.style={
            rectangle, rounded corners=3pt, draw=black!60, thick,
            fill=#1, minimum width=3.2cm, minimum height=1.0cm,
            text centered, font=\small, text width=3cm
        },
        process/.default=blue!8,
        data/.style={
            rectangle, draw=black!40, fill=gray!5,
            minimum width=2.6cm, minimum height=0.7cm,
            text centered, font=\footnotesize, text width=2.5cm
        },
        arrow/.style={->, thick, black!60},
        note/.style={font=\scriptsize\itshape, text=gray!70, text width=2.8cm},
    ]

    % ── Row 1: HOG features ──
    \node[data] (hog) {HOG descriptors\\$\{\mathbf{h}_i\}_{i=1}^N$};

    % ── Row 2: Dissimilarity ──
    \node[process=yellow!12, below=0.6cm of hog] (dissim)
        {CEMD dissimilarity\\$D(\mathbf{h}^A, \mathbf{h}^B)$};

    % ── Row 3: NFA ──
    \node[process=orange!12, below=0.6cm of dissim] (nfa)
        {NFA computation\\$\text{NLFA}(A,B)$};
    \node[note, right=0.8cm of nfa] {$\mu_A, \sigma_A$ from\\background model};

    % ── Row 4: Thresholding ──
    \node[process=red!10, below=0.6cm of nfa] (threshold)
        {Reciprocal thresholding\\$\varepsilon = 5{\times}10^{-4}$};

    % ── Row 5: Graph ──
    \node[process=green!12, below=0.6cm of threshold] (graph)
        {Similarity graph $G$\\$w_{ij} = \frac{1}{2}(\text{NLFA}_{ij} + \text{NLFA}_{ji})$};

    % ── Row 6: Leiden ──
    \node[process=blue!15, below=0.6cm of graph] (leiden)
        {Leiden algorithm\\$\gamma = 1.0$};

    % ── Row 7: Output ──
    \node[data, below=0.6cm of leiden] (partition)
        {Initial partition\\$\{C_1, \ldots, C_K\}$};

    % ── Arrows ──
    \draw[arrow] (hog) -- (dissim);
    \draw[arrow] (dissim) -- (nfa);
    \draw[arrow] (nfa) -- (threshold);
    \draw[arrow] (threshold) -- (graph);
    \draw[arrow] (graph) -- (leiden);
    \draw[arrow] (leiden) -- (partition);

    % ── Side annotations ──
    \node[note, left=0.8cm of dissim] {Per-cell circular\\EMD (Eq.~\ref{eq:cemd})};
    \node[note, left=0.8cm of threshold] {Both $\text{NLFA}(i,j)$\\and $\text{NLFA}(j,i)$\\must exceed $\tau$};
    \node[note, left=0.8cm of leiden] {RBConfiguration\\vertex partition};

    \end{tikzpicture}%
    }
    \caption{Clustering pipeline: from HOG features to an initial partition.
    Pairwise CEMD dissimilarities are converted into NFA scores under the \textit{a contrario} model, thresholded reciprocally to build a similarity graph, and partitioned by the Leiden algorithm.}
    \label{fig:clustering-pipeline}
\end{figure}



% ============================================================================
\subsection{Cluster Refinement}
\label{sec:refinement}

The Leiden partition provides an initial clustering that may contain impure clusters (multiple character types) or over-fragmented characters (a type split across multiple clusters).
A sequential refinement pipeline addresses these issues through three stages.

% ---------- Hausdorff splitting ----------
\subsubsection{Stage 1: Hausdorff-Based Cluster Splitting}
\label{sec:hausdorff-split}

Large clusters ($\ge 5$ members) are inspected for visual heterogeneity using the Hausdorff distance computed on registered binary images.

\paragraph{Image registration.}
For each pair of characters within a cluster, one is registered onto the other using a multiscale inverse compositional (IC) algorithm~\cite{briand2018ipol}.
The IC algorithm estimates a homographic transformation $T$ by minimising a Lorentzian robust error function over a Gaussian pyramid with downsampling factor $\eta = 0.5$ and minimum resolution~$32$\,px.
The Gauss--Newton optimisation uses Farid~$5 \times 5$ gradient kernels and runs for up to 5 iterations per scale with convergence threshold $10^{-3}$.

\paragraph{Hausdorff distance.}
After registration, the directed Hausdorff distance~\cite{rony2025hausdorff} between the binarised masks $M_1$ and $M_2$ is computed via distance transforms (using the \texttt{distorch} library):
\begin{equation}
    d_H(M_1, M_2) = \max\!\left(\sup_{p \in \partial M_1} d(p, \partial M_2),\; \sup_{q \in \partial M_2} d(q, \partial M_1)\right),
\end{equation}
where $\partial M$ denotes the boundary of mask $M$.
The distance computation is fully batched on GPU.

\paragraph{Hierarchical splitting.}
The pairwise Hausdorff distance matrix within each cluster is used to build a hierarchical linkage (average method).
The dendrogram is cut at a threshold $\tau_\text{split} = 21.5$: sub-trees below this distance are kept as sub-clusters.
Characters with unknown OCR labels ($\square$) are excluded from the distance computation and subsequently assigned to the largest resulting sub-cluster of their original Leiden cluster.

\begin{figure}[t]
    \centering
    \fbox{\parbox{0.95\linewidth}{\centering\vspace{3cm}%
        \textbf{[PLACEHOLDER]} --- Example dendrogram showing a cluster being split.
        Left: character patches sorted by sub-cluster.
        Right: the linkage dendrogram with the cut threshold shown as a horizontal line.
    \vspace{3cm}}}
    \caption{Hausdorff-based cluster splitting.
    The pairwise Hausdorff distance dendrogram (right) reveals two visually distinct groups.
    The cut at $\tau_\text{split} = 21.5$ (dashed line) separates them into two sub-clusters (left).}
    \label{fig:hausdorff-split}
\end{figure}


% ---------- OCR rematch ----------
\subsubsection{Stage 2: OCR-Based Rematching}
\label{sec:ocr-rematch}

Small clusters ($\le 3$ members) that result from the splitting stage or from fragmented Leiden partitions are candidates for label-based merging.
For each small cluster:
\begin{enumerate}
    \item The dominant recognised label is determined (excluding unknowns).
    \item The largest cluster sharing the same dominant label is identified.
    \item All members of the small cluster are reassigned to the target cluster.
\end{enumerate}
This purely label-driven step is fast (no GPU computation) and handles the common case where a rare glyph variant is split from its main cluster.


% ---------- PCA z-score rematch ----------
\subsubsection{Stage 3: PCA Z-Score Rematching}
\label{sec:pca-rematch}

Remaining small clusters ($\le 3$ members) without matching OCR labels are tested against candidate large clusters ($\ge 10$ members) using an image-space compatibility test:
\begin{enumerate}
    \item \textbf{Candidate selection.}  The $k = 5$ nearest large clusters are identified by $L^2$ distance between mean HOG feature vectors.
    \item \textbf{Registration.}  The query character is registered against the most central member (highest degree centrality in $G$) of each candidate cluster using IC registration.
    \item \textbf{PCA projection.}  A PCA model is built from the registered images of the candidate cluster's members (retaining $d = 5$ components).
    The registered query image is projected into this space.
    \item \textbf{Z-score test.}  For each principal component $j$, the z-score $z_j = |q_j - \mu_j| / \sigma_j$ is computed.
    If $\max_j z_j < z_\text{max} = 3.0$, the query is deemed compatible with the candidate.
    \item \textbf{Merge decision.}  Among all compatible candidates, the query is merged into the one with the lowest $\max_j z_j$.
\end{enumerate}

\begin{figure}[t]
    \centering
    \fbox{\parbox{0.95\linewidth}{\centering\vspace{3cm}%
        \textbf{[PLACEHOLDER]} --- Illustration of PCA z-score rematching.
        Show a singleton being projected into the PCA space of a candidate cluster and accepted / rejected.
    \vspace{3cm}}}
    \caption{PCA z-score rematching.
    A singleton character (red) is registered against the anchor of a candidate cluster (blue) and projected into the cluster's PCA space.
    If all z-scores are below the threshold (right panel), the singleton is merged into the cluster.}
    \label{fig:pca-rematch}
\end{figure}


% ============================================================================
\subsection{Post-Clustering Refinement}
\label{sec:post-clustering}

Two optional refinement steps may be applied after the main clustering pipeline.

\subsubsection{CHAT-Based Cluster Splitting}
\label{sec:chat-split}

Clusters whose OCR label purity falls below 90\% (i.e.\ no single label accounts for $\ge 90\%$ of the known-label members) are further split by grouping members with the same dominant CHAT label into separate sub-clusters, provided each resulting sub-cluster has $\ge 2$~members with that label.

\subsubsection{Hapax Association}
\label{sec:hapax}

Singleton clusters (``hapax'') whose CHAT confidence exceeds~$0.3$ are merged into the closest cluster sharing the same recognised label.
The distance criterion is adaptive: the dissimilarity from the hapax to the candidate cluster must not exceed the cluster's internal median dissimilarity (or a fixed threshold if specified).


% ============================================================================
\subsection{Evaluation Metrics}
\label{sec:metrics}

The quality of the clustering is evaluated against OCR-derived reference labels (excluding unknowns) using the following metrics:

\begin{itemize}
    \item \textbf{Adjusted Rand Index (ARI)}~\cite{hubert1985comparing}: chance-adjusted pairwise agreement, range $[-1, 1]$.
    Primary selection criterion for hyperparameter sweeps.
    \item \textbf{Normalised Mutual Information (NMI)}: shared information between clusterings, range $[0, 1]$.
    \item \textbf{Homogeneity / Completeness / V-measure}~\cite{rosenberg2007v}: homogeneity measures within-cluster label purity; completeness measures whether all instances of a label are in one cluster; V-measure is their harmonic mean.
    \item \textbf{Purity and Inverse Purity}: purity $= \frac{1}{N}\sum_k \max_j |c_k \cap t_j|$; inverse purity swaps the roles of clusters and classes.
    \item \textbf{Pairwise F1 score}: F1 computed from pair-level TP/FP/FN via the contingency matrix.
    \item \textbf{Accuracy (Hungarian matching)}: optimal one-to-one cluster-to-class assignment via the Hungarian algorithm.
\end{itemize}

Cluster-level diagnostics include Shannon entropy and normalised entropy of the label distribution within each cluster, as well as per-label completeness (fraction of a label's instances in its dominant cluster).

% ============================================================================
%  Refinement Pipeline Figure — three-stage post-partition refinement
% ============================================================================
\begin{figure}[t]
    \centering
    \resizebox{\linewidth}{!}{%
    \begin{tikzpicture}[
        >=stealth,
        node distance=0.3cm and 1.5cm,
        stage/.style={
            rectangle, rounded corners=4pt, draw=#1!70, thick,
            fill=#1!10, minimum width=4.5cm, minimum height=1.2cm,
            text centered, font=\small\bfseries, text width=4.2cm
        },
        detail/.style={
            rectangle, rounded corners=2pt, draw=black!30,
            fill=white, minimum width=4.0cm, minimum height=0.6cm,
            text centered, font=\scriptsize, text width=3.8cm
        },
        io/.style={
            rounded rectangle, draw=black!50, fill=gray!8,
            minimum width=3.5cm, minimum height=0.8cm,
            text centered, font=\small
        },
        arrow/.style={->, very thick, black!60},
        note/.style={font=\scriptsize\itshape, text=gray!60, text width=3.2cm, align=left},
    ]

    % ── Input ──
    \node[io] (input) {Leiden partition};

    % ── Stage 1: Hausdorff split ──
    \node[stage=red, below=0.7cm of input] (split) {Hausdorff Splitting};
    \node[detail, below=0.1cm of split] (split1) {Pairwise IC registration + Hausdorff};
    \node[detail, below=0.05cm of split1] (split2) {Average linkage dendrogram};
    \node[detail, below=0.05cm of split2] (split3) {Cut at $\tau_\text{split}$; unknowns $\to$ largest sub-cluster};

    \node[note, right=0.8cm of split] {Impure clusters\\$\to$ visually coherent\\sub-clusters};

    % ── Stage 2: OCR rematch ──
    \node[stage=blue, below=0.7cm of split3] (ocr) {OCR Rematching};
    \node[detail, below=0.1cm of ocr] (ocr1) {Small clusters ($\le 3$): find dominant label};
    \node[detail, below=0.05cm of ocr1] (ocr2) {Merge into largest cluster with same label};

    \node[note, right=0.8cm of ocr] {Fast, label-driven\\(no GPU)};

    % ── Stage 3: PCA rematch ──
    \node[stage=green!70!black, below=0.7cm of ocr2] (pca) {PCA Z-Score Rematching};
    \node[detail, below=0.1cm of pca] (pca1) {$k$-NN candidate clusters (HOG distance)};
    \node[detail, below=0.05cm of pca1] (pca2) {IC register query $\to$ anchor};
    \node[detail, below=0.05cm of pca2] (pca3) {PCA projection + z-score test};

    \node[note, right=0.8cm of pca] {Image-space\\compatibility test\\for remaining singletons};

    % ── Output ──
    \node[io, below=0.7cm of pca3] (output) {Refined clusters};

    % ── Arrows ──
    \draw[arrow] (input) -- (split);
    \draw[arrow] (split3.south) -- ++(0,-0.2) -| ([xshift=0cm]ocr.north);
    \draw[arrow] (ocr2.south) -- ++(0,-0.2) -| ([xshift=0cm]pca.north);
    \draw[arrow] (pca3.south) -- ++(0,-0.2) -| ([xshift=0cm]output.north);

    % ── Stage labels ──
    \node[left=0.6cm of split, font=\footnotesize\bfseries, text=red!70] {Stage 1};
    \node[left=0.6cm of ocr, font=\footnotesize\bfseries, text=blue!70] {Stage 2};
    \node[left=0.6cm of pca, font=\footnotesize\bfseries, text=green!50!black] {Stage 3};

    \end{tikzpicture}%
    }
    \caption{Three-stage cluster refinement pipeline.
    After the initial Leiden partition, clusters are sequentially split by visual dissimilarity (Stage~1), merged by OCR label (Stage~2), and merged by image-space PCA compatibility (Stage~3).
    Each stage is composable and produces structured diagnostics for reporting.}
    \label{fig:refinement-pipeline}
\end{figure}



% ============================================================================
\subsection{Hyperparameter Summary}
\label{sec:hyperparameters}

Table~\ref{tab:hyperparameters} summarises the key hyperparameters of the pipeline and their default values.

\begin{table}[t]
    \centering
    \small
    \caption{Key hyperparameters and their default values.}
    \label{tab:hyperparameters}
    \begin{tabular}{@{}llll@{}}
        \toprule
        \textbf{Stage} & \textbf{Parameter} & \textbf{Symbol} & \textbf{Default} \\
        \midrule
        \multirow{3}{*}{Extraction}
            & CRAFT text threshold    & $\tau_\text{text}$   & 0.6  \\
            & CRAFT mask threshold    & $\tau_\text{mask}$   & 0.3  \\
            & Merge distance          & $d_\text{min}$       & 8\,px \\
        \midrule
        \multirow{4}{*}{HOG}
            & Cell size               & ---                  & $24 \times 24$\,px \\
            & Number of bins          & $B$                  & 16 \\
            & Gradient $\sigma$       & $\sigma$             & 5 \\
            & Clip threshold          & $t$                  & 0.2 \\
        \midrule
        \multirow{2}{*}{Matching}
            & Dissimilarity metric    & ---                  & CEMD \\
            & NFA threshold           & $\varepsilon$        & $5 \times 10^{-4}$ \\
        \midrule
        \multirow{2}{*}{Clustering}
            & Reciprocal edges        & ---                  & Yes \\
            & Resolution ($\gamma$)   & $\gamma$             & 1.0 \\
        \midrule
        \multirow{2}{*}{Splitting}
            & Hausdorff threshold     & $\tau_\text{split}$  & 21.5 \\
            & Min.\ cluster size      & ---                  & 5 \\
            & Linkage method          & ---                  & Average \\
        \midrule
        \multirow{3}{*}{Rematching}
            & Max.\ small cluster     & ---                  & 3 \\
            & PCA components          & $d$                  & 5 \\
            & Z-score threshold       & $z_\text{max}$       & 3.0 \\
        \bottomrule
    \end{tabular}
\end{table}
