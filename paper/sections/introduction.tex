\section{Introduction}
\label{sec:introduction}

% TODO: Write full introduction text.

% a. Historical context: Chinese printing history, woodblock and movable type, character variation, calligraphy models.
%      Some statistics & information on Chinese characters could be included 
%      -> Characters are complex, with many strokes and intricate details, making them challenging for OCR and clustering.
%      -> This problem is especially pronounced in historical prints, where degradation and variation are common.
%      -> Character distribution is heavy-tailed, with a few common characters and many rare ones, complicating clustering and glossary construction.
%      -> The problem of reconstructing character models from historical prints is important for digital preservation, palaeographic study, font reconstruction, and cross-document comparison of printing traditions.
% b. Problem statement: Given a historical printed book, reconstruct faithful vectorised versions of the character models used to produce it, and assemble them into a character glossary.
% c. Contribution / claims
%       -> We formulate reverse typography as the problem of reconstructing a vectorised character glossary from historical prints
%       -> We show that combining a contrario statistical matching on HOG descriptors + correlation clustering (CPM/Leiden) works well, and have ablations to prove some claims
%       -> Affine scale-space vectorisation produces calligraphy-faithful beautiful SVGs
%       -> Demonstration on a challenging historical book with unique character occurrences.
%       -> We show how to use this method for when there's no ground truth bboxes / labels, by combining CRAFT+Kraken
%       -> And that, moreover, our method is resilient to OCR errors

% Note: this introduction is too long


% The structure below is a scaffold.

The history of Chinese printing spans more than twelve centuries.
Woodblock printing appeared in the 8th century: entire pages were carved in reverse on wood planks, inked, and pressed onto paper.
Movable-type printing, invented by Bi Sheng around~1040, allowed individual characters to be cast in ceramic, wood, or metal and assembled into pages.
Both technologies coexisted for more than a thousand years, each producing distinct patterns of character variation.
In every case, craftsmen imitated character models painted by a calligrapher.

\emph{Reverse typography} addresses the inverse problem: given a historical printed book, reconstruct faithful vectorised versions of the character models used to produce it.
The goal is to automatically generate a \emph{character glossary}---an inventory of all distinct character forms, each represented as a clean vector outline respecting the original calligraphy.
Such a glossary enables digital preservation, palaeographic study of printing workshops, font reconstruction, and cross-document comparison of printing traditions.

% ── Main figure: end-to-end pipeline demonstration ──
% ============================================================================
%  Main pipeline example figure — five-panel overview of stages I & II
%
%  Generated by: scripts/generate_paper_main_figure.py
%  Prerequisite: run the script to produce figures/generated/main_pipeline.pdf
% ============================================================================
\begin{figure*}[t]
    \centering
    \includegraphics[width=\textwidth]{figures/generated/main_pipeline.pdf}
    \caption{End-to-end pipeline demonstration on a page from the
    \emph{Ying huan zhi lue} (1848).
    \textbf{(a)}~Input scan.
    \textbf{(b)}~Character extraction: CRAFT detection, watershed segmentation,
    binarisation, Mahalanobis assignment, and post-filtering yield individual
    character bounding boxes.
    \textbf{(c)}~Layout analysis identifies text columns (coloured overlays)
    and assigns a right-to-left, top-to-bottom reading order.
    \textbf{(d)}~Subcolumn crops are fed to the Kraken OCR model; predictions
    are matched to CRAFT detections via the Hungarian algorithm on vertical
    distances.
    \textbf{(e)}~Final alignment: OCR predictions are compared against the
    transcription via Levenshtein edit operations.
    Green~=~match, orange~=~OCR differs from transcription,
    blue~=~OCR only (no transcription available), grey~=~unknown.}
    \label{fig:main-example}
\end{figure*}


We demonstrate the method on the \emph{Ying huan zhi lue} (Brief Records of the World, 1848), a Chinese book printed from individually carved woodblocks in which every character occurrence was uniquely engraved---a maximally challenging scenario for clustering.
The pipeline processes \nPages{} page images into approximately \nPatches{} character patches and produces a glossary of \nGlossaryEntries{} distinct character forms.
