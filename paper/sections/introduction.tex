\section{Introduction}

\label{sec:introduction}


% To-do:

% - Add a figure showing the problem and the goal (e.g., a page of the book, with character bounding boxes, and an example of the desired output: a glossary with clean vectorised characters).
% - Explain the goal in a paragraph
% - Explain the stakes and applications of reverse typography (digital preservation, palaeography, font reconstruction, cross-document comparison).
% - Discuss the challenges of historical printed documents (degraded ink, irregular spacing, archaic character variants, heavy tail of rare characters, lack of labelled data).
% - Introduce:
%   | The key contributions
%   | The overall pipeline
%   | Some result statistics?

% Here is an attempt at this exercice:
%
% The history of Chinese printing spans more than twelve centuries.
% Woodblock printing appeared in the 8th century: entire pages were carved in reverse on wood planks or ceramic plates, inked, and pressed onto paper.
% Movable-type printing, invented by Bi Sheng around~1040, allowed individual characters to be cast in ceramic, wood, or metal and assembled into pages.
% Both technologies coexisted for more than a thousand years, each producing distinct patterns of character variation.
% In every case, craftsmen carving wood or moulding metal imitated character models painted by a calligrapher.
% The variability of printed characters therefore depends on the regularity of craftsmanship, quality of paper and ink, and centuries of document ageing.

% \emph{Reverse typography} addresses the inverse problem: given a historical printed book, reconstruct faithful vectorised versions of the character models used to produce it.
% The goal is to automatically generate a \emph{character glossary}---an inventory of all distinct character forms, each represented as a clean vector outline respecting the original calligraphy.
% Such a glossary enables digital preservation, palaeographic study of printing workshops, font reconstruction, and cross-document comparison of printing traditions.
% Modern OCR systems, trained on millions of annotated examples, adapt poorly to historical prints due to degraded ink, irregular spacing, archaic character variants, and the absence of labelled training data.
% A fully unsupervised approach is therefore required.

% Our pipeline leverages three key techniques:
% \begin{enumerate}[nosep]
%     \item \textbf{CRAFT}~\cite{baek2019character}, a learned text detector that provides reliable character-level extraction without fine-tuning;
%     \item \textbf{Affine scale-space vectorisation}~\cite{alvarez1993axioms,ciomaga2017curvature}, an affine-invariant geometric PDE that smooths character boundaries into clean vector outlines while preserving their essential shape;
%     \item \textbf{A contrario matching}~\cite{desolneux2000meaningful}, a statistical framework providing a principled, adaptive test for whether two character images depict the same glyph.
% \end{enumerate}
% Characters deemed similar are connected in a graph partitioned by the Leiden community detection algorithm~\cite{traag2019louvain}, and a composable refinement pipeline fusing visual registration and linguistic signals produces the final clusters from which the glossary is assembled.

% We demonstrate the method on the \emph{Ying huan zhi lue} (Brief Records of the World), a famous 1848 Chinese book by Xu Jiyu, printed from individually carved woodblocks.
% In this book, every character occurrence was uniquely engraved---a maximally challenging scenario for clustering, since no two instances of the same character share the same printing block.
% We provide evidence that the method adapts to other historical printed books.